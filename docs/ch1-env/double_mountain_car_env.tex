对于我创建的 Double Mountain Car 环境,是仿造经典的离散动作版本的 MountainCar 问题的扩展版本,具体代码可以参考 \href{https://gymnasium.farama.org/environments/classic_control/mountain_car}{https://gymnasium.farama.org/environments/classic\_control/mountain\_car}。
其环境的详细描述如下:

\begin{definition*}{环境描述}
\# 问题描述

    设计一个包含 两个山地车 的强化学习环境。该环境是经典离散动作的 MountainCar 问题的扩展版本~\footnote{如果你不熟悉 MountainCar 问题,请参考 \href{https://gymnasium.farama.org/environments/classic_control/mountain_car}{https://gymnasium.farama.org/environments/classic\_control/mountain\_car}。},两个智能体(小车)必须协作,利用加速度在重力势能影响下成功爬坡,到达目标位置。两个小车可以分别独立控制,它们的任务是同时达到终点位置的目标状态,回合才会终止。这个问题与 MountainCar 问题几乎一致,但是可以帮助大家熟悉 gymnasium 库中的 \href{https://gymnasium.farama.org/api/spaces/fundamental/#gymnasium.spaces.Box}{spaces.Box} 和 \href{https://gymnasium.farama.org/api/spaces/fundamental/#gymnasium.spaces.MultiDiscrete}{spaces.MultiDiscrete} 这两个基础空间类。

\# 具体要求

对于这个问题的相关具体要求,可以仔细阅读并参考 MountainCar 问题的源代码 \href{https://github.com/Farama-Foundation/Gymnasium/blob/main/gymnasium/envs/classic_control/mountain_car.py}{gymnasium/envs/classic\_control/mountain\_car.py}。
我们唯一需要做的事情就是将原本单个 Car 的状态空间与动作空间变成 两个 Car 的状态与动作空间。
熟悉相关基础类,并初步入门多智能体控制。
\end{definition*}

在此基础上,我舍弃了使用 \textsf{render} 进行可视化,而是使用 \textsf{spaces.Box} 构建小车的观测环境和使用 \href{https://gymnasium.farama.org/api/spaces/fundamental/#gymnasium.spaces.MultiDiscrete}{\textsf{spaces.MultiDiscrete}} 来构建里两个小车的联合动作。
对于 \textsf{spaces.MultiDiscrete} 的使用,与 \textsf{spaces.MultiBinary} 类似,只是参数变成了 \textsf{np.array}。
例如对于任天堂游戏手柄,可以离散化成 \(3\) 个离散的动作空间:
\begin{itemize}
    \item 方向键:\(5\) 个离散的元素 - NOOP [0]、上 [1]、右 [2]、下 [3]、左 [4] - 参数:最小值:\(0\),最大值:\(4\)。
    \item 按钮 A:\(2\) 个离散的元素 - NOOP [0],按下 [1] - 参数:最小值:\(0\),最大值:\(1\)。
    \item 按钮 B:\(2\) 个离散的元素 - NOOP [0],按下 [1] - 参数:最小值:\(0\),最大值:\(1\)。
\end{itemize}
对应的空间代码为
\begin{minted}[fontsize=\small, breaklines]{python}
spaces.MultiDiscrete([ 5, 2, 2 ])
\end{minted}


而在我们的环境中,对应的是两个小车,各 \(3\) 个动作,而环境可以被视为一个简单的 \(4\)-维的 \textsf{spaces.Box}~\footnote{也可以通过 \textsf{spaces.Tuple} 和 \textsf{spaces.Box} 组合构建。} 所以动作和环境的初始化如下:
\begin{minted}[fontsize=\small, breaklines]{python}

...
# State bounds for both cars
self.low = np.array(
    [
        self.min_position,
        -self.max_speed,  # Car 1
        self.min_position,
        -self.max_speed,  # Car 2
    ],
    dtype=np.float32,
)

self.high = np.array(
    [
        self.max_position,
        self.max_speed,  # Car 1
        self.max_position,
        self.max_speed,  # Car 2
    ],
    dtype=np.float32,
)
...
self.action_space = spaces.MultiDiscrete([3, 3])
self.observation_space = spaces.Box(self.low, self.high, dtype=np.float32)
...
\end{minted}
其他设置都和原始的 MountainCar 问题一致。
而对于可视化操作,这里就没有实现。

\subsection{实现代码}

对于 Double Mountain Car 问题的环境的具体代码,请参考附录~\ref{sec:double-mountain-car-env}。
