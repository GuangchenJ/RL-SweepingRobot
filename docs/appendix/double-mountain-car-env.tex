在这部分,展示我实现的 Double Mountain Car,代码接口符合 Gymnasium 的标准。

\begin{minted}[frame=single, fontsize=\small, linenos, breaklines]{python}
import math
from typing import Optional

import gymnasium as gym
import numpy as np
from gymnasium import spaces
from gymnasium.envs.classic_control import utils


class DoubleMountainCarEnv(gym.Env):
    """
    Double Mountain Car Environment - Two cars that need to cooperate to reach the goal.

    ## Description
    This is an extension of the classic Mountain Car environment with two cars.
    Each car can be controlled independently, and they need to strategically
    accelerate to reach their respective goals.

    ## Observation Space
    The observation is a `ndarray` with shape `(4,)`:
    - [0]: position of car 1
    - [1]: velocity of car 1
    - [2]: position of car 2
    - [3]: velocity of car 2

    ## Action Space
    MultiDiscrete([3, 3]) - Each car has 3 actions:
    - 0: Accelerate to the left
    - 1: Don't accelerate
    - 2: Accelerate to the right
    """

    metadata = {
        "render_modes": [],
        "render_fps": 30,
    }

    def __init__(self, render_mode: Optional[str] = None, goal_velocity: float = 0):
        # Physical parameters
        self.min_position = -1.2
        self.max_position = 0.6
        self.max_speed = 0.07
        self.goal_position = 0.5
        self.goal_velocity = goal_velocity

        self.force = 0.001
        self.gravity = 0.0025

        # State bounds for both cars
        self.low = np.array(
            [
                self.min_position,
                -self.max_speed,  # Car 1
                self.min_position,
                -self.max_speed,  # Car 2
            ],
            dtype=np.float32,
        )

        self.high = np.array(
            [
                self.max_position,
                self.max_speed,  # Car 1
                self.max_position,
                self.max_speed,  # Car 2
            ],
            dtype=np.float32,
        )

        # Rendering
        self.render_mode = render_mode
        self.screen_width = 600
        self.screen_height = 400
        self.screen = None
        self.clock = None
        self.isopen = True

        # Action and observation spaces
        self.action_space = spaces.MultiDiscrete([3, 3])
        self.observation_space = spaces.Box(self.low, self.high, dtype=np.float32)

        # Initialize state
        self.state = None

    def step(self, action: np.ndarray):
        assert self.action_space.contains(action), (
            f"{action!r} ({type(action)}) invalid"
        )

        # Extract positions and velocities
        pos1, vel1, pos2, vel2 = self.state

        # Update car 1
        vel1 += (action[0] - 1) * self.force + math.cos(3 * pos1) * (-self.gravity)
        vel1 = np.clip(vel1, -self.max_speed, self.max_speed)
        pos1 += vel1
        pos1 = np.clip(pos1, self.min_position, self.max_position)
        if pos1 == self.min_position and vel1 < 0:
            vel1 = 0

        # Update car 2
        vel2 += (action[1] - 1) * self.force + math.cos(3 * pos2) * (-self.gravity)
        vel2 = np.clip(vel2, -self.max_speed, self.max_speed)
        pos2 += vel2
        pos2 = np.clip(pos2, self.min_position, self.max_position)
        if pos2 == self.min_position and vel2 < 0:
            vel2 = 0

        # Check termination - both cars need to reach the goal
        car1_at_goal = pos1 >= self.goal_position and vel1 >= self.goal_velocity
        car2_at_goal = pos2 >= self.goal_position and vel2 >= self.goal_velocity
        terminated = bool(car1_at_goal and car2_at_goal)

        # Reward: -1 per timestep, bonus when both reach goal
        reward = -1.0
        # Update state
        self.state = np.array([pos1, vel1, pos2, vel2], dtype=np.float32)

        if self.render_mode == "human":
            self.render()

        return self.state, reward, terminated, False, {}

    def reset(
        self,
        *,
        seed: Optional[int] = None,
        options: Optional[dict] = None,
    ):
        super().reset(seed=seed)

        # Initialize both cars at random positions
        low, high = utils.maybe_parse_reset_bounds(options, -0.6, -0.4)

        # Car 1 starts at a random position
        pos1 = self.np_random.uniform(low=low, high=high)
        vel1 = 0

        # Car 2 starts at a different random position
        pos2 = self.np_random.uniform(low=low, high=high)
        vel2 = 0

        self.state = np.array([pos1, vel1, pos2, vel2], dtype=np.float32)

        if self.render_mode == "human":
            self.render()

        return self.state, {}
\end{minted}
