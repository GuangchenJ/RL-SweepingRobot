在这部分,展示我实现的 Double Mountain Car 环境的智能体 \(Q\)-learning 实现代码。

\begin{minted}[frame=single, fontsize=\small, linenos, breaklines]{python}
from pathlib import Path
from typing import Tuple

import gymnasium as gym
import matplotlib.pyplot as plt
import numpy as np
import pandas as pd
from gymnasium.wrappers import TimeLimit
from stable_baselines3 import PPO
from stable_baselines3.common.evaluation import evaluate_policy
from stable_baselines3.common.monitor import Monitor
from stable_baselines3.common.vec_env import DummyVecEnv, VecNormalize

# -----------------------------------------------------------------------------
# 0.  Helper wrappers
# -----------------------------------------------------------------------------

COMBOS: Tuple[Tuple[int, int], ...] = (
    (0, 0),
    (0, 1),
    (0, 2),
    (1, 0),
    (1, 1),
    (1, 2),
    (2, 0),
    (2, 1),
    (2, 2),
)


class NineToMultiAction(gym.ActionWrapper):
    def __init__(self, env: gym.Env):
        super().__init__(env)
        self.action_space = gym.spaces.Discrete(9)

    def action(self, act: int):
        return COMBOS[act]


class DenseRewardWrapper(gym.Wrapper):
    """Dense shaping: add avg Δx × weight per step."""

    def __init__(self, env: gym.Env, weight: float = 2.0):
        super().__init__(env)
        self.weight = weight
        self._last_pos = None

    def reset(self, **kw):
        obs, info = self.env.reset(**kw)
        self._last_pos = np.array([obs[0], obs[2]])
        return obs, info

    def step(self, act):
        obs, r, term, trunc, info = self.env.step(act)
        pos = np.array([obs[0], obs[2]])
        shaped = r + self.weight * (pos - self._last_pos).mean()
        self._last_pos = pos
        return obs, shaped, term, trunc, info


# -----------------------------------------------------------------------------
# 1.  I/O paths
# -----------------------------------------------------------------------------

RES_DIR = Path("res") / "double_mountain_car"
RES_DIR.mkdir(parents=True, exist_ok=True)
MODEL_PATH = RES_DIR / "ppo_double_mountain_car.zip"
VECNORM_PATH = RES_DIR / "vecnorm.pkl"

MAX_EPISODE_STEPS = 400

# -----------------------------------------------------------------------------
# 2.  Environment factory
# -----------------------------------------------------------------------------


def make_env() -> gym.Env:
    from double_mountain_car_env import DoubleMountainCarEnv

    base = DoubleMountainCarEnv(render_mode=None)
    base = NineToMultiAction(base)
    base = DenseRewardWrapper(base)
    base = TimeLimit(base, max_episode_steps=MAX_EPISODE_STEPS)
    return Monitor(base, str(RES_DIR))


# -----------------------------------------------------------------------------
# 3.  Training
# -----------------------------------------------------------------------------

vec_env = DummyVecEnv([make_env])
vec_env = VecNormalize(vec_env, norm_obs=True, norm_reward=True, clip_obs=10.0)

model = PPO(
    "MlpPolicy",
    vec_env,
    verbose=1,
    n_steps=1024,
    batch_size=512,
    gamma=0.99,
    learning_rate=3e-4,
    clip_range=0.2,
    device="cpu",
)

TOTAL_STEPS = 1_000_000
print(f"Training for {TOTAL_STEPS:,} steps …\n")
model.learn(total_timesteps=TOTAL_STEPS, progress_bar=True)
model.save(MODEL_PATH)
vec_env.save(str(VECNORM_PATH))
print("Training finished. Model & VecNormalize saved.\n")


# -------------------------------------------------------------------
# 4.  Plot helper (raw + EMA $\alpha$=0.9)
# -------------------------------------------------------------------
def compute_ema(arr, alpha: float = 0.9):
    """EMA, 支持 list / ndarray / Series,返回 ndarray"""
    arr = np.asarray(arr, dtype=float)
    if arr.size == 0:
        return arr
    ema = np.empty_like(arr)
    ema[0] = arr[0]
    for i in range(1, arr.size):
        ema[i] = alpha * ema[i - 1] + (1 - alpha) * arr[i]
    return ema


def plot_series(y, title: str, ylabel: str, fname: str):
    y = np.asarray(y, dtype=float)
    x = np.arange(len(y))
    ema = compute_ema(y, alpha=0.95)
    plt.figure(figsize=(8, 5), dpi=120)
    plt.plot(x, y, alpha=0.3, label="raw")
    plt.plot(x, ema, label="EMA $\alpha$=0.95", lw=2)
    plt.xlabel("Episode")
    plt.ylabel(ylabel)
    plt.title(title)
    plt.legend()
    plt.tight_layout()
    plt.savefig(RES_DIR / fname)
    plt.close()


# -------------------------------------------------------------------
# 5.  Plot metrics from monitor.csv
# -------------------------------------------------------------------
monitor_file = RES_DIR / "monitor.csv"
if monitor_file.exists():
    df = pd.read_csv(monitor_file, skiprows=1)

    if "r" in df:
        plot_series(df["r"], "Episode Reward", "Reward", "reward_curve.png")

    if "l" in df:
        plot_series(df["l"], "Episode Length", "Steps", "episode_length.png")

        # 成功率:步数 < TimeLimit → 1,否则 0
        success = (df["l"] < MAX_EPISODE_STEPS).astype(int)
        plot_series(
            compute_ema(success),
            "Success Rate",
            "Rate",
            "success_rate.png",
        )

# -----------------------------------------------------------------------------
# 6.  Evaluation only block (optional)
# -----------------------------------------------------------------------------


def evaluate(trials: int = 20):
    raw_eval = DummyVecEnv([make_env])
    eval_env = VecNormalize.load(str(VECNORM_PATH), raw_eval)
    model_ = PPO.load(str(MODEL_PATH), env=eval_env, device="cpu")

    mean_r, std_r = evaluate_policy(
        model_,
        eval_env,
        n_eval_episodes=trials,
        deterministic=True,
    )
    print(f"Mean reward over {trials} episodes: {mean_r:.2f} ± {std_r:.2f}")


if __name__ == "__main__":
    evaluate()

\end{minted}