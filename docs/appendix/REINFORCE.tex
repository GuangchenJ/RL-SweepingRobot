在这部分,展示我实现的 REINFORCE 策略梯度算法代码,所用环境是扫地机器人。

\begin{minted}[frame=single, fontsize=\small, linenos, breaklines]{python}
import os

import matplotlib.pyplot as plt
import numpy as np
import torch
import torch.nn as nn
import torch.nn.functional as F
import torch.optim as optim
from torch.distributions import Categorical
from torch.optim.lr_scheduler import StepLR

from sweeping_robot_env import SweepingRobotEnv


# Policy Network 定义
class PolicyNetwork(nn.Module):
    def __init__(self, input_size, hidden_size, output_size):
        super(PolicyNetwork, self).__init__()
        self.fc1 = nn.Linear(input_size, hidden_size)
        self.fc2 = nn.Linear(hidden_size, hidden_size)
        self.fc3 = nn.Linear(hidden_size, output_size)

        # 初始化权重
        nn.init.xavier_uniform_(self.fc1.weight)
        nn.init.xavier_uniform_(self.fc2.weight)
        nn.init.xavier_uniform_(self.fc3.weight)

        nn.init.zeros_(self.fc1.bias)
        nn.init.zeros_(self.fc2.bias)
        nn.init.zeros_(self.fc3.bias)

    def forward(self, x):
        x = F.relu(self.fc1(x))
        x = F.relu(self.fc2(x))
        x = self.fc3(x)
        # 添加数值稳定性
        # x = torch.clamp(x, min=-10, max=10)  # 防止过大的 logits
        return F.softmax(x, dim=-1)


def state_to_tensor(state, size):
    """将状态转换为 one-hot 编码的张量"""
    row, col = state
    state_vector = torch.zeros(size * size)
    state_vector[row * size + col] = 1
    return state_vector


def compute_returns(rewards, gamma):
    """计算折扣回报"""
    returns = []
    running_return = 0
    # 反向遍历所有的收益,计算折现收益
    for r in reversed(rewards):
        running_return = r + gamma * running_return
        returns.insert(0, running_return)
    return torch.tensor(returns, dtype=torch.float32)


def compute_ema(data, alpha=0.9):
    """计算指数移动平均(EMA)"""
    ema = []
    if len(data) > 0:
        ema.append(data[0])
        for i in range(1, len(data)):
            ema.append(alpha * ema[-1] + (1 - alpha) * data[i])
    return ema


def plot_training_results(
    episode_rewards,
    episode_lengths,
    episode_success_flags,
    save_path="training_results_with_success.png",
):
    """绘制训练奖励、步数和成功率曲线"""
    episodes = list(range(len(episode_rewards)))

    # 计算EMA
    ema_rewards = compute_ema(episode_rewards, alpha=0.99)
    ema_lengths = compute_ema(episode_lengths, alpha=0.99)
    success_rates = [
        np.mean(episode_success_flags[max(0, i - 99) : i + 1]) * 100 for i in episodes
    ]
    ema_success = compute_ema(success_rates, alpha=0.95)

    fig, axs = plt.subplots(3, 1, figsize=(12, 15))

    # 奖励
    axs[0].plot(episodes, episode_rewards, "b-", alpha=0.3, label="Reward")
    axs[0].plot(episodes, ema_rewards, "r-", linewidth=2, label="EMA Reward ($\alpha$=0.99)")
    axs[0].set_title("Episode Total Reward")
    axs[0].set_xlabel("Episode")
    axs[0].set_ylabel("Reward")
    axs[0].legend()
    axs[0].grid(True, alpha=0.3)

    # 步数
    axs[1].plot(episodes, episode_lengths, "g-", alpha=0.3, label="Steps")
    axs[1].plot(
        episodes, ema_lengths, "orange", linewidth=2, label="EMA Steps ($\alpha$=0.99)"
    )
    axs[1].set_title("Episode Length")
    axs[1].set_xlabel("Episode")
    axs[1].set_ylabel("Steps")
    axs[1].legend()
    axs[1].grid(True, alpha=0.3)

    # 成功率
    axs[2].plot(episodes, success_rates, "c-", alpha=0.3, label="Success Rate (100-ep)")
    axs[2].plot(episodes, ema_success, "m-", linewidth=2, label="EMA Success ($\alpha$=0.95)")
    axs[2].set_title("Success Rate")
    axs[2].set_xlabel("Episode")
    axs[2].set_ylabel("Success Rate (%)")
    axs[2].legend()
    axs[2].grid(True, alpha=0.3)

    plt.tight_layout()
    plt.savefig(save_path, dpi=300, bbox_inches="tight")
    # plt.show()

    # 打印最终统计
    print("\n训练统计信息:")
    print(f"最高奖励: {max(episode_rewards):.2f}")
    print(f"最后 100 个 episode 平均奖励: {np.mean(episode_rewards[-100:]):.2f}")
    print(f"最后 100 个 episode 平均步数: {np.mean(episode_lengths[-100:]):.2f}")
    print(
        f"最后 100 个 episode 成功率: {np.mean(episode_success_flags[-100:]) * 100:.2f}%"
    )


def visualize_policy(env, policy_net, save_path="policy_visualization.png"):
    """可视化学习到的策略"""
    import matplotlib.image as mpimg
    import matplotlib.pyplot as plt
    import numpy as np
    import torch

    fig, ax = plt.subplots(1, 1, figsize=(10, 10))

    # 动作箭头的方向(调整上下方向)
    action_arrows = {
        0: (0, -0.4),  # 上:在倒序显示中,上变成了向下
        1: (0, 0.4),  # 下:在倒序显示中,下变成了向上
        2: (-0.4, 0),  # 左:保持不变
        3: (0.4, 0),  # 右:保持不变
    }

    # 载入图标
    trash_img = mpimg.imread("icons/trash.png")
    charger_img = mpimg.imread("icons/charger.png")
    block_img = mpimg.imread("icons/block.png")

    # 绘制网格
    for i in range(env.size + 1):
        ax.axhline(y=i, color="gray", linewidth=0.5)
        ax.axvline(x=i, color="gray", linewidth=0.5)

    # 为每个状态绘制最优动作
    for row in range(env.size):
        for col in range(env.size):
            pos = np.array([row, col])
            center_x = col + 0.5
            center_y = row + 0.5  # 直接使用row,不再倒序

            # 图片的extent也需要相应调整
            img_extent = (col, col + 1, row, row + 1)

            # 特殊位置标记
            if np.array_equal(pos, env._trash_location):
                ax.imshow(trash_img, extent=img_extent)
                continue
            elif np.array_equal(pos, env._charging_station_location):
                ax.imshow(charger_img, extent=img_extent)
                continue
            elif np.array_equal(pos, env._obstacle_location):
                ax.imshow(block_img, extent=img_extent)
                continue

            # 获取该状态的动作概率
            state_tensor = state_to_tensor((row, col), env.size).unsqueeze(0)
            with torch.no_grad():
                action_probs = policy_net(state_tensor).squeeze().numpy()

            # 绘制动作箭头(根据概率调整透明度)
            for action, prob in enumerate(action_probs):
                if prob > 0.1:  # 只显示概率大于 0.1 的动作
                    dx, dy = action_arrows[action]
                    ax.arrow(
                        center_x,
                        center_y,
                        dx * prob,
                        dy * prob,
                        head_width=0.1,
                        head_length=0.1,
                        fc="blue",
                        alpha=prob,
                    )

    ax.set_xlim(0, env.size)
    ax.set_ylim(0, env.size)
    ax.set_aspect("equal")
    ax.set_title(
        "The visualization of learned strategies\n(Arrow => Action, Transparency => Probability)"
    )
    ax.set_xlabel("col")
    ax.set_ylabel("row")

    plt.tight_layout()
    plt.savefig(save_path, dpi=300, bbox_inches="tight")


def create_folder(path):
    # 判断文件夹是否存在
    if not os.path.exists(path):
        # 如果文件夹不存在,则创建
        os.makedirs(path)


def train_policy_gradient(env, policy_net, optimizer, episodes, gamma=0.99):
    """使用 REINFORCE 算法训练策略网络"""
    episode_rewards = []
    episode_lengths = []
    success_count = 0
    episode_success_flags = []

    # 用于存储最近的成功轨迹
    recent_successes = []

    print("环境信息:")
    print(f"- 垃圾位置: {env._trash_location}")
    print(f"- 充电站位置: {env._charging_station_location}")
    print(f"- 障碍物位置: {env._obstacle_location}\n")

    for episode in range(episodes):
        # 收集一个episode的数据
        states = []
        actions = []
        rewards = []

        obs, _ = env.reset()
        done = False
        total_reward = 0
        steps = 0

        # 动态调整最大步数
        max_steps = min(100, 50 + episode // 20)  # 随着训练逐渐增加最大步数

        while not done and steps < max_steps:  # 添加最大步数限制
            # 将状态转换为张量
            state_tensor = state_to_tensor(obs, env.size).unsqueeze(0)

            # 获取动作概率分布
            action_probs = policy_net(state_tensor)

            # 采样动作
            m = Categorical(action_probs)
            action = m.sample()

            # 执行动作
            next_obs, reward, terminated, truncated, _ = env.step(action.item())

            # 记录数据
            states.append(state_tensor)
            actions.append(action)
            rewards.append(reward)

            obs = next_obs
            total_reward += reward
            steps += 1
            done = terminated or truncated

        # 检查是否成功到达垃圾
        if total_reward >= 5:
            success_count += 1
            recent_successes.append(
                {
                    "states": states.copy(),
                    "actions": actions.copy(),
                    "rewards": rewards.copy(),
                }
            )
            # 只保留最近的成功轨迹
            if len(recent_successes) > 10:
                recent_successes.pop(0)
            episode_success_flags.append(1)
        else:
            episode_success_flags.append(0)

        episode_rewards.append(total_reward)
        episode_lengths.append(steps)

        # 计算回报
        returns = compute_returns(rewards, gamma)

        # 标准化回报(只有当有多个步骤时才标准化),使用 baseline 减少方差
        if len(returns) > 1:
            baseline = returns.mean()
            advantages = returns - baseline
            advantages = (advantages - advantages.mean()) / (advantages.std() + 1e-8)
        else:
            # 如果只有一步,不进行标准化
            advantages = returns

        # 计算损失并更新网络
        policy_loss = []
        entropy_bonus = 0
        for state, action, advantage in zip(states, actions, advantages):
            action_probs = policy_net(state)
            m = Categorical(action_probs)
            policy_loss.append(-m.log_prob(action) * advantage)
            # 动态调整熵正则化系数,随训练进度递减(从 0.4 降到 0.01)
            # 早期阶段鼓励更多探索,后期减少探索
            entropy_coef = max(0.1, 0.8 * (1 - episode / episodes))
            entropy_bonus += m.entropy() * entropy_coef

        # 如果有成功经验,偶尔从成功轨迹中学习(经验回放)
        if len(recent_successes) > 0 and episode % 50 == 0:
            # 随机选择一个成功轨迹
            success_traj = recent_successes[np.random.randint(len(recent_successes))]
            success_returns = compute_returns(success_traj["rewards"], gamma)

            if len(success_returns) > 1:
                success_baseline = success_returns.mean()
                success_advantages = success_returns - success_baseline
                success_advantages = (
                    success_advantages - success_advantages.mean()
                ) / (success_advantages.std() + 1e-8)
            else:
                success_advantages = success_returns

            # 添加成功轨迹的损失
            for state, action, advantage in zip(
                success_traj["states"], success_traj["actions"], success_advantages
            ):
                action_probs = policy_net(state)
                m = Categorical(action_probs)
                policy_loss.append(
                    -m.log_prob(action) * advantage * 0.5
                )  # 降低权重避免过拟合

        # 更新策略网络
        if len(policy_loss) > 0:
            optimizer.zero_grad()
            loss = (
                torch.stack(policy_loss).sum() - entropy_bonus
            )  # 减去熵奖励(鼓励探索)

            # 检查损失是否为 NaN
            if torch.isnan(loss):
                print(f"警告:Episode {episode} 出现 NaN 损失,跳过更新")
                continue

            loss.backward()

            # 梯度裁剪,防止梯度爆炸
            torch.nn.utils.clip_grad_norm_(policy_net.parameters(), max_norm=5.0)

            optimizer.step()
            # 学习率每 1000 个 episode 衰减 5%
            scheduler.step()

        # 打印进度
        if episode % 100 == 0:
            avg_reward = np.mean(episode_rewards[-100:])
            avg_length = np.mean(episode_lengths[-100:])
            success_rate = success_count / (episode + 1) * 100
            recent_success_rate = (
                sum(1 for r in episode_rewards[-100:] if r >= 5)
                / min(100, len(episode_rewards))
                * 100
            )
            print(
                f"Episode {episode}, Average Reward (last 100): {avg_reward:.2f}, "
                f"Average Length: {avg_length:.1f}, Success Rate: {success_rate:.1f}%, "
                f"Recent Success Rate (100 steps): {recent_success_rate:.1f}%"
            )

    return episode_rewards, episode_lengths, episode_success_flags


if __name__ == "__main__":
    create_folder("res/sweep_robot/policy-gradient")

    # 创建环境
    env = SweepingRobotEnv(render_mode="rgb_array", size=5)

    # 创建策略网络》
    input_size = env.size * env.size  # one-hot编码的状态大小
    hidden_size = 64
    output_size = env.action_space.n

    policy_net = PolicyNetwork(input_size, hidden_size, output_size)
    # RMSProp 在策略梯度方法中通常效果更好,能够更平滑地更新参数
    # optimizer = optim.RMSprop(policy_net.parameters(), lr=0.001, alpha=0.99, eps=1e-8)
    # 使用 Adam 优化器,通常比 RMSprop 更稳定
    optimizer = optim.Adam(policy_net.parameters(), lr=0.001)
    # 每 1000 个 step(可以等价视为 episode)将学习率乘以 0.95
    scheduler = StepLR(optimizer, step_size=1000, gamma=0.95)

    # 训练参数
    episodes = 10000
    gamma = 0.995

    print("开始训练Policy Gradient...")
    episode_rewards, episode_lengths, episode_success_flags = train_policy_gradient(
        env, policy_net, optimizer, episodes, gamma
    )

    # 绘制训练结果
    print("\n绘制训练结果...")
    plot_training_results(
        episode_rewards,
        episode_lengths,
        episode_success_flags,
        save_path="res/sweep_robot/policy-gradient/training_results.png",
    )

    # 可视化学习到的策略
    print("\n可视化学习到的策略...")
    visualize_policy(
        env,
        policy_net,
        save_path="res/sweep_robot/policy-gradient/policy_visualization.png",
    )

    # 测试训练好的策略
    print("\n测试训练好的策略...")
    test_episodes = 2000
    test_rewards = []
    test_success = 0

    for i in range(test_episodes):
        obs, _ = env.reset()
        done = False
        total_reward = 0
        steps = 0
        path = [obs]

        while not done and steps < 50:  # 限制最大步数防止死循环
            state_tensor = state_to_tensor(obs, env.size).unsqueeze(0)
            with torch.no_grad():
                action_probs = policy_net(state_tensor)
                action = torch.argmax(action_probs, dim=1).item()

            obs, reward, terminated, truncated, _ = env.step(action)
            path.append(obs)
            total_reward += reward
            done = terminated or truncated
            steps += 1

        test_rewards.append(total_reward)
        if total_reward >= 5:
            test_success += 1
        print(f"\r测试Episode {i + 1}: 总奖励 = {total_reward}, 步数 = {steps}")

    print(f"\n测试平均奖励: {np.mean(test_rewards):.2f}")
    print(f"测试成功率: {test_success / test_episodes * 100:.1f}%")

    # 保存模型
    torch.save(
        policy_net.state_dict(),
        "res/sweep_robot/policy-gradient/policy_gradient_model.pth",
    )
    print("模型已保存为 'res/sweep_robot/policy-gradient/policy_gradient_model.pth'")

    env.close()

\end{minted}
