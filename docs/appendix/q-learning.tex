在这部分,展示我实现的对于多开关匹配环境的智能体 \(Q\)-learning 实现代码。

\begin{minted}[frame=single, fontsize=\small, linenos, breaklines]{python}
import os
from collections import deque

import matplotlib.pyplot as plt
import numpy as np

from multi_switch_env import MultiSwitchEnv


class TrainingMonitor:
    """训练监控器,记录各种指标用于绘图"""

    def __init__(self):
        self.episode_rewards = []
        self.episode_lengths = []
        self.q_value_mean = []
        self.q_value_max = []
        self.success_rate = deque(maxlen=20)  # 滑动窗口记录最近 20 个 episode 的成功率
        self.epsilon_values = []

    def record_episode(self, total_reward, episode_length, success):
        self.episode_rewards.append(total_reward)
        self.episode_lengths.append(episode_length)
        self.success_rate.append(1.0 if success else 0.0)

    def record_q_values(self, q_table):
        self.q_value_mean.append(np.mean(q_table))
        self.q_value_max.append(np.max(q_table))

    def record_epsilon(self, epsilon):
        self.epsilon_values.append(epsilon)

    def compute_ema(self, data, alpha=0.9):
        """计算指数加权移动平均(EMA)"""
        ema = np.zeros_like(data, dtype=float)
        if len(data) > 0:
            ema[0] = data[0]
            for i in range(1, len(data)):
                ema[i] = alpha * ema[i - 1] + (1 - alpha) * data[i]
        return ema

    def plot_results(self, save_path="/multi_switch/training_results.png"):
        """绘制训练结果的综合图表"""
        fig, axes = plt.subplots(2, 3, figsize=(15, 10))
        fig.suptitle("Multi-Switch Environment Training Metrics", fontsize=16)

        # 设置统一的EMA参数
        ema_alpha = 0.95

        # 1. Episode Rewards
        ax1 = axes[0, 0]
        ax1.plot(
            self.episode_rewards,
            alpha=0.3,
            color="blue",
            linewidth=1,
            label="Raw Rewards",
        )
        # 添加EMA平滑线
        if len(self.episode_rewards) > 1:
            ema_rewards = self.compute_ema(self.episode_rewards, ema_alpha)
            ax1.plot(ema_rewards, "b-", linewidth=2.5, label=f"EMA ($\alpha$={ema_alpha})")
        ax1.set_xlabel("Episode")
        ax1.set_ylabel("Total Reward")
        ax1.set_title("Episode Rewards Over Time")
        ax1.legend()
        ax1.grid(True, alpha=0.3)

        # 2. Episode Lengths
        ax2 = axes[0, 1]
        ax2.plot(self.episode_lengths, "g-", alpha=0.3, linewidth=1, label="Raw Steps")
        if len(self.episode_lengths) > 1:
            ema_lengths = self.compute_ema(self.episode_lengths, ema_alpha)
            ax2.plot(ema_lengths, "g-", linewidth=2.5, label=f"EMA ($\alpha$={ema_alpha})")
        ax2.set_xlabel("Episode")
        ax2.set_ylabel("Steps")
        ax2.set_title("Episode Length (Steps to Complete)")
        ax2.legend()
        ax2.grid(True, alpha=0.3)

        # 3. Success Rate
        ax3 = axes[0, 2]
        if len(self.success_rate) > 0:
            success_rates = []
            for i in range(len(self.episode_rewards)):
                end_idx = min(i + 1, len(self.success_rate))
                start_idx = max(0, end_idx - 20)
                if end_idx > start_idx:
                    rate = sum(list(self.success_rate)[start_idx:end_idx]) / (
                        end_idx - start_idx
                    )
                    success_rates.append(rate * 100)
            ax3.plot(
                success_rates, "c-", alpha=0.3, linewidth=1, label="20-Episode Window"
            )
            if len(success_rates) > 1:
                ema_success = self.compute_ema(success_rates, ema_alpha)
                ax3.plot(ema_success, "b-", linewidth=2.5, label=f"EMA ($\alpha$={ema_alpha})")
        ax3.set_xlabel("Episode")
        ax3.set_ylabel("Success Rate (%)")
        ax3.set_title("Success Rate")
        ax3.legend()
        ax3.grid(True, alpha=0.3)
        ax3.set_ylim([0, 105])

        # 4. Q-Value Mean
        ax4 = axes[1, 0]
        ax4.plot(self.q_value_mean, "purple", alpha=0.3, linewidth=1, label="Raw Mean")
        if len(self.q_value_mean) > 1:
            ema_q_mean = self.compute_ema(self.q_value_mean, ema_alpha)
            ax4.plot(
                ema_q_mean, color="purple", linewidth=2.5, label=f"EMA ($\alpha$={ema_alpha})"
            )
        ax4.set_xlabel("Episode")
        ax4.set_ylabel("Mean Q-Value")
        ax4.set_title("Average Q-Value Over Time")
        ax4.legend()
        ax4.grid(True, alpha=0.3)

        # 5. Q-Value Max
        ax5 = axes[1, 1]
        ax5.plot(self.q_value_max, "orange", alpha=0.3, linewidth=1, label="Raw Max")
        if len(self.q_value_max) > 1:
            ema_q_max = self.compute_ema(self.q_value_max, ema_alpha)
            ax5.plot(
                ema_q_max, color="orange", linewidth=2.5, label=f"EMA ($\alpha$={ema_alpha})"
            )
        ax5.set_xlabel("Episode")
        ax5.set_ylabel("Max Q-Value")
        ax5.set_title("Maximum Q-Value Over Time")
        ax5.legend()
        ax5.grid(True, alpha=0.3)

        # 6. Epsilon Decay
        ax6 = axes[1, 2]
        if self.epsilon_values:
            ax6.plot(
                self.epsilon_values, "red", alpha=0.3, linewidth=1, label="Raw Epsilon"
            )
            if len(self.epsilon_values) > 1:
                ema_epsilon = self.compute_ema(self.epsilon_values, ema_alpha)
                ax6.plot(
                    ema_epsilon, "red", linewidth=2.5, label=f"EMA ($\alpha$={ema_alpha})"
                )
            ax6.set_xlabel("Episode")
            ax6.set_ylabel("Epsilon")
            ax6.set_title("Exploration Rate (Epsilon) Decay")
            ax6.legend()
            ax6.grid(True, alpha=0.3)

        plt.tight_layout()
        plt.savefig(save_path, dpi=300, bbox_inches="tight")

    def plot_q_heatmap(self, q_table, episode, save_path="q_heatmap.png"):
        """绘制Q表的热力图(仅适用于小规模Q表)"""
        # 将多维Q表展平为2D用于可视化
        flat_shape = (np.prod(q_table.shape[:3]), np.prod(q_table.shape[3:]))
        q_flat = q_table.reshape(flat_shape)

        plt.figure(figsize=(10, 8))
        plt.imshow(q_flat, cmap="coolwarm", aspect="auto")
        plt.colorbar(label="Q-Value")
        plt.title(f"Q-Table Heatmap at Episode {episode}")
        plt.xlabel("Action Combinations")
        plt.ylabel("State Combinations")
        plt.tight_layout()
        plt.savefig(save_path, dpi=300, bbox_inches="tight")
        plt.close()


def visualize_top_actions(
    q_table,
    target_state,
    num_switches=3,
    top_k=3,
    save_path="res/multi_switch/top_actions_visualization.png",
):
    """
    对于每一个可能的状态,绘制其对应Q值最高的 top_k 个动作。
    并在标题中标注目标状态。
    """
    state_space = [list(s) for s in np.ndindex(*(2,) * num_switches)]

    rows, cols = 2, 4  # 设置子图分布为 2 行 4 列
    fig, axes = plt.subplots(rows, cols, figsize=(cols * 4, rows * 4))
    axes = axes.flatten()

    for ax, state in zip(axes, state_space):
        state_idx = tuple(state)
        q_values = q_table[state_idx]

        flat_qs = q_values.reshape(-1)
        top_action_indices = flat_qs.argsort()[-top_k:][::-1]  # 降序取前k
        top_q_values = flat_qs[top_action_indices]
        top_actions = [
            np.unravel_index(idx, q_values.shape) for idx in top_action_indices
        ]

        labels = ["".join(map(str, a)) for a in top_actions]

        ax.bar(range(top_k), top_q_values)
        ax.set_xticks(range(top_k))
        ax.set_xticklabels(labels, rotation=45)
        ax.set_title(
            f"State: {''.join(map(str, state))}\nTarget: {''.join(map(str, target_state))}"
        )
        ax.set_ylabel("Q-Value")

    # 删除多余的子图
    for ax in axes[len(state_space) :]:
        ax.axis("off")

    plt.tight_layout()
    plt.suptitle("Top Actions per State", fontsize=16, y=1.05)
    plt.savefig(save_path, dpi=300, bbox_inches="tight")
    plt.close()


def create_folder(path):
    # 判断文件夹是否存在
    if not os.path.exists(path):
        # 如果文件夹不存在,则创建
        os.makedirs(path)


if __name__ == "__main__":
    create_folder("res/multi_switch")

    # 设置随机种子以便复现
    np.random.seed(42)

    env = MultiSwitchEnv(render_mode="human", num_switches=3)
    q_table = np.zeros([2] * 3 + [2] * 3)  # Q-table shape: (2,2,2,2,2,2)

    # 超参数
    alpha = 0.1  # 学习率
    gamma = 0.95  # 折扣因子
    epsilon_start = 0.9  # 初始探索率
    epsilon_end = 0.01  # 最终探索率
    epsilon_decay = 0.995  # 探索率衰减
    episodes = 1000  # 增加训练轮数以更好观察曲线

    # 创建训练监控器
    monitor = TrainingMonitor()

    # 训练循环
    epsilon = epsilon_start

    for episode in range(episodes):
        obs, _ = env.reset()
        done = False
        truncated = False
        total_reward = 0
        steps = 0

        print(f"\nEpisode {episode + 1}/{episodes} ($\epsilon$={epsilon:.3f})")

        while not done and not truncated:
            obs_idx = tuple(obs)

            # $\epsilon$-贪婪策略
            if np.random.rand() < epsilon:
                action = env.action_space.sample()
            else:
                action = np.unravel_index(
                    np.argmax(q_table[obs_idx]), q_table[obs_idx].shape
                )

            next_obs, reward, done, truncated, _ = env.step(np.array(action))
            next_obs_idx = tuple(next_obs)

            # Q-learning 更新
            best_next = np.max(q_table[next_obs_idx])
            q_table[obs_idx + tuple(action)] += alpha * (
                reward + gamma * best_next - q_table[obs_idx + tuple(action)]
            )

            obs = next_obs
            total_reward += reward
            steps += 1

            if env.render_mode == "human":
                env.render()

        # 记录训练数据
        monitor.record_episode(total_reward, steps, done)
        monitor.record_q_values(q_table)
        monitor.record_epsilon(epsilon)

        # 衰减探索率
        epsilon = max(epsilon_end, epsilon * epsilon_decay)

        print(
            f"\nTotal Reward: {total_reward:.2f} | Steps: {steps} | Success: {'Yes' if done else 'No'}"
        )

        # 每 100 个 episode 保存一次Q表热力图
        if (episode + 1) % 100 == 0:
            monitor.plot_q_heatmap(
                q_table,
                episode + 1,
                f"res/multi_switch/q_heatmap_episode_{episode + 1}.png",
            )

    env.close()

    # 绘制所有训练结果
    print("\n正在生成训练结果图表...")
    monitor.plot_results("res/multi_switch/multiswitch_training_results.png")

    # 打印最终统计信息
    print("\n=== 训练完成 ===")
    print(f"最终成功率: {np.mean(list(monitor.success_rate)) * 100:.1f}%")
    print(f"最后 10 轮平均奖励: {np.mean(monitor.episode_rewards[-10:]):.2f}")
    print(f"Q 表平均值: {np.mean(q_table):.4f}")
    print(f"Q 表最大值: {np.max(q_table):.4f}")

    # 测试训练好的策略
    print("\n=== 测试最终策略 (贪婪策略) ===")
    test_episodes = 10
    test_rewards = []

    for i in range(test_episodes):
        obs, _ = env.reset()
        done = False
        truncated = False
        total_reward = 0

        while not done and not truncated:
            obs_idx = tuple(obs)
            # 使用纯贪婪策略
            action = np.unravel_index(
                np.argmax(q_table[obs_idx]), q_table[obs_idx].shape
            )
            obs, reward, done, truncated, _ = env.step(np.array(action))
            total_reward += reward

        test_rewards.append(total_reward)
        print(
            f"测试 {i + 1}: 奖励 = {total_reward:.2f}, 成功 = {'是' if done else '否'}"
        )

    print(f"\n测试平均奖励: {np.mean(test_rewards):.2f}")
    print(f"测试成功率: {sum(r > 0 for r in test_rewards) / test_episodes * 100:.0f}%")

    visualize_top_actions(
        q_table, target_state=env.target_state, num_switches=3, top_k=3
    )

\end{minted}