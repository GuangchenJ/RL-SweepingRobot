在这部分,展示我实现的多开关匹配的环境代码,代码接口符合 Gymnasium 的标准。

\begin{minted}[frame=single, fontsize=\small, linenos, breaklines]{python}
import sys
import time

import gymnasium as gym
import numpy as np
from gymnasium import spaces


class MultiSwitchEnv(gym.Env):
    metadata = {"render_modes": ["human"], "render_fps": 10}

    def __init__(self, render_mode=None, num_switches=3):
        super().__init__()
        self.num_switches = num_switches

        # 状态空间和动作空间都是 MultiBinary: 每个开关两个状态 0/1
        self.observation_space = spaces.MultiBinary(num_switches)
        self.action_space = spaces.MultiBinary(num_switches)  # 每个开关切或不切

        self.target_state = np.random.randint(0, 2, size=num_switches)
        self.state = None
        self.render_mode = render_mode
        self.steps = 0
        self.max_steps = 5

    def reset(self, seed=None, options=None):
        super().reset(seed=seed)
        self.state = self.np_random.integers(0, 2, size=self.num_switches)
        self.steps = 0
        return self.state.copy(), {}

    def step(self, action):
        self.steps += 1

        # 应用动作: 切换值(异或操作)
        self.state = (self.state + action) % 2

        done = np.array_equal(self.state, self.target_state)
        reward = 1.0 if done else -0.1
        truncated = self.steps >= self.max_steps

        return self.state.copy(), reward, done, truncated, {}

    def render(self):
        output = f"\rCurrent State: {self.state} | Target: {self.target_state}"
        sys.stdout.write(output)
        sys.stdout.flush()
        time.sleep(0.3)

    def close(self):
        pass
\end{minted}
