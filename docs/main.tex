\documentclass[citestyle=gb7714-2015, bibstyle=gb7714-2015,lang=cn,14pt,scheme=chinese]{elegantbook}

\title{深度学习模型的学习动态}
\subtitle{xxxxx}

\author{姜广琛}
\institute{西北工业大学}
\date{2025/04/17}
\version{0.1}
% \bioinfo{自定义}{信息}

% \extrainfo{注意:本模板自 2023 年 1 月 1 日开始,不再更新和维护!}

\setcounter{tocdepth}{3}

\logo{logo-blue.png}
\cover{cover.jpg}

% 本文档命令




% 修改标题页的橙色带
\definecolor{customcolor}{RGB}{32,178,170}
\colorlet{coverlinecolor}{customcolor}
\usepackage{cprotect}

\addbibresource[location=local]{reference.bib} % 参考文献,不要删除
\ExecuteBibliographyOptions{sorting=ynt}

\begin{document}

% \maketitle
% \frontmatter

% \tableofcontents

\mainmatter%

\chapter{学习动态(Learning Dynamics)}

学习动态(Learning Dynamics)通常是一个总称,用于描述特定的因素如何影响模型预测。
但在本文中,我们跟随论文 Learning Dynamics of {LLM} Finetuning~\cite{DBLP:conf/iclr/RenS25} 的思路,探究“how the change in model's parameter \(\bm{\theta}\) influences the corresponding change in \(f_{\bm{\theta}}\)”,也就是 \(\textcolor{cyan}{\triangle \bm{\theta}}\) 和 \(\textcolor{orange}{\triangle f_{\bm{\theta}}}\) 之间的关系。

但在这里,我们需要复习一下利用梯度下降(gradient descent, GD)更新模型参数的过程。

\section{梯度下降(Gradient Descent)简介}

在这一部分,我们来一步步讲解“模型利用梯度下降更新参数”的过程。我们会从最基础的迭代计算开始,逐步过渡到机器学习和深度学习中的应用,并结合公式帮助你理解。

\subsection{从基础的迭代计算开始}

假设我们有一个我们有一个函数 \(f_{\bm{\theta}} \left( \cdot \right)\),我们希望找到使这个函数取得最小值的 \(\bm{\theta}\) 值。
一般来说都会使用梯度下降法,其核心思想包括:
\begin{itemize}
    \item 计算当前点的梯度 \(\nabla f_{\bm{\theta}}\):梯度表示函数在该点的变化率,指明函数增长最快的方向。
    \item 沿着梯度的反方向更新参数:因为我们希望最小化函数值,所以要朝着函数下降最快的方向前进。
    \item 更新参数:重复以上两步,直到收敛通过迭代不断调整 \(\bm{\theta}\) 的值,使函数值逐步减小,直到达到最小值。
\end{itemize}
更新公式如下:
\begin{equation}\label{eq:gradient_descent_iterative_update}
    \triangle \bm{\theta} \coloneq \bm{\theta}^{t+1} - \bm{\theta}^{t} = - \eta \nabla f_{\bm{\theta}}
\end{equation}
其中,\(\alpha\) 是学习率,控制每次更新的步长。\(\nabla f_{\bm{\theta}}\) 是具有参数函数 \(\bm{\theta}\) 的函数 \(f_{\bm{\theta}} \left(x\right)\) 在点 \(\bm{\theta}\) 处的梯度。

\subsection{机器学习中的梯度下降}

在机器学习与深度学习中,我们的目标是训练一个模型,使其在给定输入 \(\bm{\mathsf{x}}\) 时,输出预测值 \(\hat{\bm{\mathsf{y}}}\),并尽可能接近真实值 \(\bm{\mathsf{y}}\)。
此时,函数 \(f_{\bm{\theta}}\) 本身是模型的预测函数,我们不能直接使用模型预测函数 \(f_{\bm{\theta}}\) 的梯度来优化模型参数,因为 \(f_{\bm{\theta}}\) 的梯度 \(\eta \nabla f_{\bm{\theta}}\) 反映的是输出对输入的敏感度,而不是模型性能的衡量。
所以,我们需要引入一个损失函数(loss function) \(\mathcal{L} \left( f_{\bm{\theta}} \left( \bm{\mathsf{x}} \right), \bm{\mathsf{y}} \right) \),衡量预测值与真实值之间的差距。
所以公式 (\ref{eq:gradient_descent_iterative_update}) 可以改写为:
\begin{equation}\label{eq:GD-loss_iterative_update}
    \triangle \bm{\theta} \coloneq \bm{\theta}^{t+1} - \bm{\theta}^{t} = - \eta \left[ \nabla_{\bm{\theta}} \mathcal{L} \left( f_{\bm{\theta}} \left( \bm{\mathsf{x}} \right), \bm{\mathsf{y}} \right) \right]^{\top}
\end{equation}
设 \(\bm{\theta} \in \mathbb{R}^{D}\),则 \(\mathcal{L} \left( f_{\bm{\theta}} \left( \bm{\mathsf{x}} \right), \bm{\mathsf{y}} \right) \in \mathbb{R}\),并且 \(\nabla_{\bm{\theta}} \mathcal{L} \left( f_{\bm{\theta}} \left( \bm{\mathsf{x}} \right), \bm{\mathsf{y}} \right) \in \mathbb{R}^{1 \times D}\)。

\subsection{常见的损失函数}

在最小化损失函数中,一般会有三种比较核心的优化算法:
\begin{itemize}
    \item \textbf{随机梯度下降(Stochastic Gradient Descent, SGD}:每次迭代只使用一个样本来计算梯度,更新速度快,但可能会有较大的波动。
    \item \textbf{批量梯度下降(Batch Gradient Descent)}:每次迭代使用整个训练集来计算梯度,收敛稳定,但计算开销大。
    \item \textbf{小批量梯度下降(Mini-batch Gradient Descent)}:每次迭代使用一小部分样本来计算梯度,兼顾了速度和稳定性,是最常用的方式。
\end{itemize}

接下来,我们以批量梯度下降为例,展示总共 \(N\) 个采样对集合 \(\left( \bm{\mathsf{X}}, \bm{\mathsf{Y}} \right) = \left\{ \left( \bm{\mathsf{x}}^{(1)}, \bm{\mathsf{y}}^{(1)} \right), \left( \bm{\mathsf{x}}^{(2)}, \bm{\mathsf{y}}^{(2)} \right), \dots, \left( \bm{\mathsf{x}}^{(N)}, \bm{\mathsf{y}}^{(N)} \right) \right\}\) 的损失函数。

\subsubsection{回归任务}

例如,回归任务中的均方误差(Mean Squared Error, MSE):
\[
    \mathcal{L} \left( f_{\bm{\theta}} \left( \bm{\mathsf{X}} \right), \bm{\mathsf{Y}} \right) = \frac{1}{N} \sum_{i=1}^{N} \left( \hat{\bm{\mathsf{y}}}^{(i)} - \bm{\mathsf{y}}^{(i)} \right)^2
\]
其中 \(f_{\bm{\theta}} \left( \bm{\mathsf{x}}^{(i)} \right) = \hat{\bm{\mathsf{y}}}^{(i)}\),均方误差(MSE)可以理解为最小化预测值和实际值之间的欧几里得距离(Euclidean distance)的平方。

\subsubsection{分类任务}

或者分类任务中的交叉熵(Cross-Entropy)损失函数:
\[
    \mathcal{L} \left( f_{\bm{\theta}} \left( \bm{\mathsf{X}} \right), \bm{\mathsf{Y}} \right) = -\frac{1}{N} \sum_{i=1}^{N} \left( \bm{\mathsf{y}}^{(i)} \log \hat{\bm{\mathsf{y}}}^{(i)} + \left( 1 - \bm{\mathsf{y}}^{(i)} \right) \log \left( 1 - \hat{\bm{\mathsf{y}}}^{(i)} \right) \right)
\]
多分类(一共 \(L\) 个类) :
\begin{equation}\label{eq:cross_entropy_loss}
    \mathcal{L} \left( f_{\bm{\theta}} \left( \bm{\mathsf{X}} \right), \bm{\mathsf{Y}} \right) = -\frac{1}{N} \sum_{i=1}^{N} \sum_{j=1}^{L} y^{(i)}_{j} \log \hat{y}^{(i)}_{j}
\end{equation}
\(y^{(i)}_{j}\) 是样本 \(\bm{\mathsf{x}}^{(i)}\) 的 one-hot 编码中第 \(j\) 个类别的标签(\(0\) 或 \(1\)),\(\hat{y}^{(i)}_{j}\) 是模型对样本 \(\bm{\mathsf{x}}^{(i)}\) 的预测是第 \(j\) 个分类的概率。

\rcomment[\textbf{为什么交叉熵是这种形式?}]{
在此,我们需要了解 \textbf{为什么交叉熵是这种形式}?
从机器学习角度看,\textbf{最小化模型损失函数的交叉熵等价于最大化概率模型的最大似然估计(Maximum Likelihood Estimation,MLE)}。

首先,我们要了解:\textbf{什么是似然(likelihood)}?
似然,你可以理解为是指在给定模型参数的情况下,观察到数据的概率有多大。

我们再回来看我们要解决的问题:\textbf{怎样调整参数 \(\bm{\theta}\) 最大化整个数据集上模型预测的联合概率(似然)}:
\[
    \prod_{i=1}^{N} P \left( \bm{\mathsf{y}}^{(i)} \mid \bm{\mathsf{x}}^{(i)}; \bm{\theta} \right)
\]
翻译成人话就是 \textbf{我们希望通过调整参数 \(\bm{\theta}\),使得最大化模型预测的输出 \(\hat{\bm{\mathsf{y}}}^{(i)}\) 是每个样本的真实标签 \(\bm{\mathsf{y}}^{(i)}\) 的概率}。
接着我们可以对两侧取对数,得到:
\[
    \sum_{i=1}^{N} \log P \left( \bm{\mathsf{y}}^{(i)} \mid \bm{\mathsf{x}}^{(i)}; \bm{\theta} \right)
\]

注意,我们以多分类任务为例,模型输出的每个类别的概率为:
\[
    \hat{\bm{\mathsf{y}}} = \left[ \hat{y}_{1}, \hat{y}_{2}, \dots, \hat{y}_{L} \right], \quad \sum_{j=1}^{L} \hat{y}_{j} = 1
\]

通常真实标签 \(\bm{\mathsf{y}}\) 是 one-hot 编码的形式:
\[
    \bm{\mathsf{y}} = \left[ y_{1}, y_{2}, \dots, y_{L} \right], \quad y_{i} \in \left\{ 0, 1 \right\}, \quad \sum_{j=1}^{L} y_{j} = 1
\]
那么对于每个样本 \(\left(\bm{\mathsf{x}}, \bm{\mathsf{y}} \right)\) 的对数似然是:
\[
    \log P \left( \bm{\mathsf{y}} \mid \bm{\mathsf{x}}; \bm{\theta} \right) = \sum_{j=1}^{L} y_{j} \log \hat{y}_{j}
\]

考虑所有的样本,并且 one-hot 编码的特性,最大化最大似然估计目标就变成了:
\[
    \underset{\bm{\theta}}{\arg \max} \sum_{i=1}^{N} \sum_{j=1}^{L} {y}^{(i)}_{j} \log \hat{y}^{(i)}_{j}  
\]

加一个负号,我们就得到了交叉熵损失函数(最小化的目标):
\[
    \underset{\bm{\theta}}{\arg \min} - \sum_{i=1}^{N} \sum_{j=1}^{L} {y}^{(i)}_{j} \log \hat{y}^{(i)}_{j}
\]
这就是我们交叉熵(Cross-Entropy)的 loss 函数的来源。
}

\section{学习动态的定义}

与梯度下降中的定义类似,一般来说我们的深度学习模型都是通过梯度下降来进行参数更新。
这里我们以一个样本的随机梯度下降(SGD)为例进行展示,在第 \(t \to t+1\) 次迭代中,模型参数 \(\bm{\theta}_{t}\) 在样本对(sample pair) \(\left( \textcolor{cyan}{\bm{\mathsf{x}}_{u}}, \textcolor{cyan}{\bm{\mathsf{y}}_{u}} \right)\) 的影响下以 \(\eta\) 的学习率更新,我们可以通过公式将这一环节描述为:
\begin{equation}\label{eq:SGD-loss_iterative_update}
    \triangle \bm{\theta} \coloneq \bm{\theta}^{t+1} - \bm{\theta}^{t} = - \eta \left[ \nabla_{\bm{\theta}} \mathcal{L} \left( f_{\bm{\theta}} \left( \textcolor{cyan}{\bm{\mathsf{x}}_{u}} \right), \textcolor{cyan}{\bm{\mathsf{y}}_{u}} \right) \right]^{\top} ; \quad \triangle f_{\bm{\theta}} \left( \textcolor{orange}{\bm{\mathsf{x}}_{o}} \right) \coloneq f_{\bm{\theta}^{t+1}} \left( \textcolor{orange}{\bm{\mathsf{x}}_{o}} \right) - f_{\bm{\theta}^{t}} \left( \textcolor{orange}{\bm{\mathsf{x}}_{o}} \right)
\end{equation}

\begin{remark}
    这里的 \(\textcolor{cyan}{u}\) 代表 update,\(\textcolor{orange}{o}\) 代表 output。
\end{remark}

简单来说,Ren et al.~\cite{DBLP:conf/iclr/RenS25} 处理的问题为:

\begin{tcolorbox} 
    \textit{After an GD update on \textcolor{cyan}{\(\bm{\mathsf{x}}_{u}\)}, how does the model's prediction on \textcolor{orange}{\(\bm{\mathsf{x}}_{o}\)} change?}
\end{tcolorbox}

我们先从一个标准的监督学习问题入手,作为热身。
在这个问题中,模型的任务是学习如何将输入 \(\bm{\mathbf{x}}\) 映射到预测输出 \(\bm{\mathbf{y}} = \left( y_{1}, y_{2}, \dots, y_{L} \right) \in \mathcal{V}^{L} \)。
这里 \(\mathcal{V}\) 是一个大小为 \(V\) 的“词汇表”(vocabulary),也就是所有可能输出的单词、符号或标签的集合。

\rcomment[\textbf{举个例子}]{
    \begin{itemize}
        \item \textbf{文本生成任务}
    \end{itemize}

    假设我们让模型生成一句英文句子,比如:
    \[
        \texttt{I love cats.}
    \]
    我们的模型的词汇表可能长这样:
    \[
        \mathcal{V} = \left\{ \texttt{<PAD>}, \texttt{<START>}, \text{<END>}, \texttt{I}, \texttt{love}, \texttt{cats}, \texttt{dogs}, \texttt{eat}, \texttt{sleep}, \cdots \right\}
    \]
    我们希望的模型输出是:
    \[
        \bm{\mathsf{y}} = \left( y_{1}, y_{2}, \dots, y_{6} \right) = \left( \texttt{<START>}, \texttt{I}, \texttt{love}, \texttt{cats}, \texttt{.}, \texttt{<END>} \right)
    \]
    这个序列的长度 \(L=6\)。
    这里 vocabulary 的大小我们可以假定是 \(V=10,000\),代表有 \(10,000\) 个单词(包括 \(\texttt{<PAD>}, \texttt{<START>}, \text{<END>}\) 等特殊标记)。

    \begin{itemize}
        \item \textbf{分类任务(以 MNIST 数据集为例)}
    \end{itemize}

    输入 \(\bm{\mathsf{x}} \in \mathbb{R}^{784}\) 是一个 \(28 \times 28 = 784\) 维的向量,表示一张手写数字的图像。
    而输出的序列的长度 \(L=1\),也就是单个标签的分类结果。
    在这里,vocabulary 指的是所有可能的数字标签 \(\mathcal{V} = \left\{ 0, 1, \dots, 0 \right\}\),\(V=10\)。
}

模型通常会先生成一个叫做 logits 的矩阵 \(\bm{\mathsf{z}} = h_{\bm{\theta}} \left( \bm{\mathsf{x}} \right) \in \mathbb{R}^{V \times L}\),是具有参数 \(\bm{\theta}\) 的模型对输入 \(\bm{\mathsf{x}}\) 的输出结果,这里是一个未能归一化得分的矩阵,称为 logits。
对于文本生成任务,其中每一列代表一句话中的一个位置的预测,每一行代表一个词汇表中的单词。
而对于分类任务,你可以理解为每一列代表对于一张图片的预测结果,每一行代表一个可能的数字标签。

接下来,模型会对这个 logits 矩阵的每一列 \(\bm{\mathsf{z}}^{(i)}\) 进行 \texttt{Softmax} 操作,得到一个概率分布:
\begin{equation}\label{eq:softmax}
    \mathtt{Softmax} \left( \bm{\mathsf{z}}^{(i)} \right)_{j} = \hat{y}^{(i)}_{j} = \frac{\exp \left( z^{(i)}_{j} \right)}{\sum_{k=1}^{V} \exp \left( z^{(i)}_{k} \right)}; \quad i = 1, 2, \dots, V
\end{equation}
其中,\(z^{(i)}_{j}\) 是第 \(i\) 列 \(\bm{\mathsf{z}}^{(i)}\) 中的第 \(j\) 个分量。

我们可以通过观察 \(\log \pi_{\bm{\theta}} \left( \bm{\mathsf{y}} \mid \bm{\mathsf{x}} \right)\),即给定输入 \(\bm{\mathsf{x}}\) 的情况下,预测输出 \(\bm{\mathsf{y}}\) 的概率,来跟踪模型的置信度(预测的结果是正确的概率)的变化。

\subsection{逐步影响分解(Per-step influence decomposition)}

我们可以定义公式 (\ref{eq:SGD-loss_iterative_update}) 对应的学习动态(learning dynamic)为:
\begin{equation}\label{eq:SGD-learning_dynamic}
    \triangle \log \pi_{\bm{\theta}^{t}} \left( \bm{\mathsf{y}} \mid \textcolor{orange}{\bm{\textsf{x}_{o}}} \right) \coloneq \log \pi_{\bm{\theta}^{t+1}} \left( \bm{\mathsf{y}} \mid \textcolor{orange}{\bm{\textsf{x}_{o}}} \right) - \log \pi_{\bm{\theta}^{t}} \left( \bm{\mathsf{y}} \mid \textcolor{orange}{\bm{\textsf{x}_{o}}} \right)
\end{equation}

为了简单,作者首先选择了 MNIST 数据集作为例子,那么当然此时为了简化,假定 \(L=1\)~\footnote{对于 \(L > 1\) 的情况也是成立的,详细可以参考证明中的步骤进行扩展,进行简单的 \(L\) 次加法就可以。}。
Ren et al.~\cite{DBLP:conf/iclr/RenS25} 提到了,这个内容是对 Ren et al.~\cite{DBLP:conf/iclr/RenGS22} 的一个结果版本。

% 为了简化表示和符号简洁,这里将 \(\pi_{\bm{\theta}^{t}}\) 用 \(\pi^{t}\) 来代替表示。

\begin{proposition}\label{prop:one_step_dynamics_decompose}
令 \(\bm{\pi}^{t} = \mathtt{Softmax} \left( \bm{\mathsf{z}}^{t} \right)\) 并且 \(\bm{\mathsf{z}}^{t} = h_{\bm{\theta}} \left( \bm{\mathsf{x}} \right)\)。一步学习动态可以被分解为:
\begin{equation}\label{eq:SGD-decomposition}
\begin{aligned}
    &\underbrace{\nabla_{\bm{\theta}} \log \pi_{\bm{\theta}^{t}} \left( \bm{\mathsf{y}} \mid \textcolor{orange}{\bm{\mathsf{x}}_{o}} \right) \mid_{\bm{\theta}^{t}}}_{\mathbb{R}^{V \times D}} \cdot \underbrace{\left( \bm{\theta}^{t+1} - \bm{\theta}^{t} \right)}_{\mathbb{R}^{D \times 1}} \\
    &= - \eta \mathcal{A}^{t} \left( \textcolor{orange}{\bm{\mathsf{x}}_{o}} \right) \mathcal{K}^{t} \left( \textcolor{orange}{\bm{\mathsf{x}}_{o}}, \textcolor{cyan}{\bm{\mathsf{x}}_{u}} \right) \mathcal{G}^{t} \left( \textcolor{cyan}{\bm{\mathsf{x}}_{u}}, \textcolor{cyan}{\bm{\mathsf{y}}_{u}} \right) + \bigO \left( \eta^{2} \left\| \left[ \nabla_{\bm{\theta}} \bm{\mathsf{z}}^{t} \left( \textcolor{cyan}{\bm{\mathsf{x}}_{u}} \right) \mid_{\bm{\theta}^{t}}  \right]^{\top} \right\|^{2}_{\texttt{op}} \right)
\end{aligned}
\end{equation}
其中,\( \mathcal{A}^{t} \left( \textcolor{orange}{\bm{\mathsf{x}}_{o}} \right) = \nabla_{\bm{\mathsf{z}}} \log \bm{\pi}^{t} \left( \textcolor{orange}{\bm{\mathsf{x}}_{o}} \right) \mid_{\bm{\mathsf{z}}^{t}} = \mathbf{I} - \mathbf{J}_{V \times 1} \left[ \bm{\pi}^{t} \left( \textcolor{orange}{\bm{\mathsf{x}}_{o}} \right) \right]^{\top} \),其中 \(\mathbf{J}_{V \times 1}\) 代表大小为 \(V \times 1\) 的全 \(1\) 矩阵;\(\mathcal{K}^{t} \left( \textcolor{orange}{\bm{\mathsf{x}}_{o}}, \textcolor{cyan}{\bm{\mathsf{x}}_{u}} \right) = \nabla_{\bm{\theta}} \log \bm{\mathsf{z}}^{t} \left( \textcolor{orange}{\bm{\mathsf{x}}_{o}} \right) \mid_{\bm{\mathsf{z}}^{t}} \cdot \left( \nabla_{\bm{\theta}} \bm{\mathsf{z}}^{t} \left( \textcolor{cyan}{\bm{\mathsf{x}}_{u}} \right) \mid_{\bm{\theta}^{t}} \right)^{\top}\) 是 logit 网络 \(\bm{\mathsf{z}}^{t}\) 的经验神经切线核(the empirical  neural tangent kernel);\(\mathcal{G}^{t} \left( \textcolor{cyan}{\bm{\mathsf{x}}_{u}}, \textcolor{cyan}{\bm{\mathsf{y}}_{u}} \right) = \left[ \nabla_{\bm{\mathsf{z}}} \mathcal{L} \left( \textcolor{cyan}{\bm{\mathsf{x}}_{u}}, \textcolor{cyan}{\bm{\mathsf{y}}_{u}} \right) \mid_{\bm{\mathsf{z}}^{t}} \right]^{\top}\)。
\end{proposition}

在上述分解中,令 \(\bm{\pi}^{t} \left( \textcolor{orange}{\bm{\mathsf{x}}_{o}} \right)\) 代表 \(\pi_{\bm{\theta}^{t}} \left( \bm{\mathsf{y}} \mid \textcolor{orange}{\bm{\mathsf{x}}_{o}} \right)\)。
如果使用 \(\bm{\pi}^{t} \left( \textcolor{orange}{\bm{\mathsf{x}}_{o}} \right) = \left[ \pi_{1}^{t}, \dots, \pi_{V}^{t} \right]^{\top}\) 代表模型在不同维度的预测概率。
并且我们已知 \(\bm{\pi}^{t}\) 是对 logits 向量~\footnote{此时 \(L=1\),logits 矩阵就变成了 logits 向量。} \(\bm{\mathsf{z}}^{t} = \left[ z_{1}^{t}, \dots, z_{V}^{t} \right]^{\top}\) 进行 \texttt{Softmax} 操作,归一化得到的概率分布:
\[
\begin{aligned}
    \log \pi_{i}^{t} &= \log \frac{\exp \left( z_{i}^{t} \right)}{\sum_{k}^{V} \exp \left( z_{k}^{t} \right)} \\
    &= z_{i}^{t} - \log \left( \sum_{k}^{V} \exp \left( z_{k}^{t} \right) \right)
\end{aligned}
\]
所以:
\[
    \frac{\partial \log \pi_{i}^{t}}{\partial z_{k}^{t}} = \delta_{i,k} - \pi_{k}^{t}
\]
其中,\(\delta_{i,k}\) 是 Kronecker delta 函数,当 \(i=k\) 时为 \(1\),否则为 \(0\)。
所以,我们可以将 \(\mathcal{A}^{t} \left( \textcolor{orange}{\bm{\mathsf{x}}_{o}} \right)\) 进一步分解为:
\begin{equation}
    \mathcal{A}^{t} \left( \textcolor{orange}{\bm{\mathsf{x}}_{o}} \right) = \nabla_{\bm{\mathsf{z}}} \log \bm{\pi}^{t} \left( \textcolor{orange}{\bm{\mathsf{x}}_{o}} \right) \mid_{\bm{\mathsf{z}}^{t}} = \mathbf{I} - \mathbf{J}_{V \times 1} \left[ \bm{\pi}^{t} \left( \textcolor{orange}{\bm{\mathsf{x}}_{o}} \right) \right]^{\top} = 
    \left[            
    \begin{array}{cccc} 
        1 - \pi_{1}^{t} & \pi_{1}^{t} & \cdots & \pi_{1}^{t} \\
        - \pi_{2}^{t} & 1 - \pi_{2}^{t} & \cdots & - \pi_{2}^{t} \\
        \vdots & \cdots & \ddots & \vdots \\
        - \pi_{V}^{t} & - \pi_{V}^{t} & \cdots & 1 - \pi_{V}^{t} \\
    \end{array}
    \right]
\end{equation}

对于 \(\mathcal{K}^{t} \left( \textcolor{orange}{\bm{\mathsf{x}}_{o}}, \textcolor{cyan}{\bm{\mathsf{x}}_{u}} \right)\),其为 \(\textcolor{orange}{\bm{\mathsf{x}}_{o}}\) 和 \(\textcolor{cyan}{\bm{\mathsf{x}}_{u}}\) 两者梯度的积。
直观的说,如果它们的梯度方向相似,则这个矩阵的 Frobenius 范数很大,反之亦然。
这个矩阵被称为经验神经切线核,可以随着网络的“相似性(similarity)”概念的发展而在训练中发生变化。
对于以非常小的学习率训练的适当初始化的非常宽(very wide)的网络,\(\mathcal{K}^{t}\) 在训练中几乎保持不变,其收敛到的 kernel 被称为 神经切线内核(neural tangent kernel)~\cite{DBLP:conf/nips/AroraDH0SW19,DBLP:conf/nips/JacotHG18}。
一般来说,在理论分析中 \(\mathcal{K}^{t}\) 通常假设不变,但作者认为这个假设太强了,所以在这里只假设是相对稳定的。

\begin{proof}[命题~\ref{prop:one_step_dynamics_decompose}]
我们想知道:模型在“观察样本(observing example)” \(\textcolor{orange}{\bm{\mathsf{x}}_{o}}\) 上的预测(\(\textcolor{orange}{\hat{\bm{\mathsf{y}}}}\))在一次参数更新(\(\bm{\theta}^{t} \to \bm{\theta}^{t+1}\))后会怎么变化?
首先对公式 (\ref{eq:SGD-learning_dynamic}) 在点 \(\bm{\theta}^{t}\) 处进行一阶泰勒展开(first-order Taylor expansion):
\begin{equation}\label{eq:SGD-learning_dynamic_taylor_expansion}
\begin{aligned}
    \log \pi_{\bm{\theta}^{t+1}} \left( \bm{\mathsf{y}} \mid \textcolor{orange}{\bm{\mathsf{x}}_{o}} \right) &= \log \pi_{\bm{\theta}^{t}} \left( \bm{\mathsf{y}} \mid \textcolor{orange}{\bm{\mathsf{x}}_{o}} \right) + \nabla_{\bm{\theta}} \log \pi_{\bm{\theta}^{t}} \left( \bm{\mathsf{y}} \mid \textcolor{orange}{\bm{\mathsf{x}}_{o}} \right) \left( \bm{\theta}^{t+1} - \bm{\theta}^{t} \right) \\
    & \qquad + \frac{1}{2} \left( \bm{\theta}^{t+1} - \bm{\theta}^{t} \right)^{\top} \nabla_{\bm{\theta}}^{2} \log \pi_{\xi} \left( \bm{\mathsf{y}} \mid \textcolor{orange}{\bm{\mathsf{x}}_{o}} \right) \left( \bm{\theta}^{t+1} - \bm{\theta}^{t} \right)
\end{aligned}
\end{equation}
其中 \(\xi\) 是介于 \(\bm{\theta}^{t}\) 和 \(\bm{\theta}^{t+1}\) 之间的某个点~\footnote{这里可以参考 Taylor 展开的 Lagrange 形式的余项}。
其上界与其最大奇异值有关,由谱范数(Spectral Norm)确定 \(\lambda_{\max}\) 来表示,下界则与其最小奇异值有关~\footnote{可以参考 谱范数(Spectral Norm)的性质:二次型的上界由谱范数确定,即 \(\bm{x}^{\top} \mathbf{H} \bm{x} \leq \left\| \mathbf{H} \right\|_{\texttt{op}}, \left\| \mathbf{H} \right\|_{\texttt{op}} = \underset{\left\|x\right\| = 1}{\sup} \bm{x}^{\top} \mathbf{H} \bm{x}\) 。并且这个上下界与最大最小特征值的关系可以参考 Rayleigh qutient,这个性质对于 complex Hermitian matrix 都成立,且实对称矩阵(在这里为 Hessian 矩阵)也是 Hermitian matrix 的一个特例。}:
\[
    \underset{\bm{\theta}^{t} \neq \bm{\theta}^{t+1}}{\sup} \left| \left( \bm{\theta}^{t+1} - \bm{\theta}^{t} \right)^{\top} \nabla_{\bm{\theta}}^{2} \log \pi_{\xi} \left( \bm{\mathsf{y}} \mid \textcolor{orange}{\bm{\mathsf{x}}_{o}} \right) \left( \bm{\theta}^{t+1} - \bm{\theta}^{t} \right) \right| = \left\| \nabla_{\bm{\theta}}^{2} \log \pi_{\xi} \left( \bm{\mathsf{y}} \mid \textcolor{orange}{\bm{\mathsf{x}}_{o}} \right) \right\|_{\texttt{op}} \cdot \left\| \bm{\theta}^{t+1} - \bm{\theta}^{t} \right\|^{2}
\]
其中,\(\left\| \cdot \right\|^{2}_{\texttt{op}}\) 是谱范数(Spectral Norm),定义为矩阵的最大奇异值。

所以:
\begin{equation}\label{eq:SGD-learning_dynamic_taylor_expansion_bigO}
\begin{aligned}
    \log \pi_{\bm{\theta}^{t+1}} \left( \bm{\mathsf{y}} \mid \textcolor{orange}{\bm{\mathsf{x}}_{o}} \right) &= \log \pi_{\bm{\theta}^{t}} \left( \bm{\mathsf{y}} \mid \textcolor{orange}{\bm{\mathsf{x}}_{o}} \right) + \nabla_{\bm{\theta}} \log \pi_{\bm{\theta}^{t}} \left( \bm{\mathsf{y}} \mid \textcolor{orange}{\bm{\mathsf{x}}_{o}} \right) \left( \bm{\theta}^{t+1} - \bm{\theta}^{t} \right) + \bigO \left( \left\| \bm{\theta}^{t+1} - \bm{\theta}^{t} \right\|^{2} \right)
\end{aligned}
\end{equation}
这里的 \(O \left( \left\| \bm{\theta}^{t+1} - \bm{\theta}^{t} \right\|^{2} \right)\) 表示不超过 \(\left\| \bm{\theta}^{t+1} - \bm{\theta}^{t} \right\|^{2}\) 这个量级。

并且,在 \(L=1\) 时, \(\nabla_{\bm{\theta}} \log \pi_{\bm{\theta}^{t}} \left( \bm{\mathsf{y}} \mid \textcolor{orange}{\bm{\mathsf{x}}_{o}} \right) \left( \bm{\theta}^{t+1} - \bm{\theta}^{t} \right) = \left\langle \nabla_{\bm{\theta}} \log \pi_{\bm{\theta}^{t}} \left( \bm{\mathsf{y}} \mid \textcolor{orange}{\bm{\mathsf{x}}_{o}} \right), \bm{\theta}^{t+1} - \bm{\theta}^{t} \right\rangle\),所以公式 (\ref{eq:SGD-learning_dynamic_taylor_expansion}) 就可以表示为 Ren et al.~\cite{DBLP:conf/iclr/RenS25} 工作中的原文形式:
\begin{equation}\label{eq:SGD-learning_dynamic_taylor_expansion_bigO_vector}
\begin{aligned}
    \log \pi_{\bm{\theta}^{t+1}} \left( \bm{\mathsf{y}} \mid \textcolor{orange}{\bm{\mathsf{x}}_{o}} \right) &= \log \pi_{\bm{\theta}^{t}} \left( \bm{\mathsf{y}} \mid \textcolor{orange}{\bm{\mathsf{x}}_{o}} \right) + \left\langle \nabla_{\bm{\theta}} \log \pi_{\bm{\theta}^{t}} \left( \bm{\mathsf{y}} \mid \textcolor{orange}{\bm{\mathsf{x}}_{o}} \right), \bm{\theta}^{t+1} - \bm{\theta}^{t} \right\rangle + \bigO \left( \left\| \bm{\theta}^{t+1} - \bm{\theta}^{t} \right\|^{2} \right)
\end{aligned}
\end{equation}
然后我们重新排列公式 (\ref{eq:SGD-learning_dynamic_taylor_expansion_bigO}),得到公式 (\ref{eq:SGD-learning_dynamic}) 中的 Talyor 展开形式:
\[
\begin{aligned}
    \triangle \log \pi_{\bm{\theta}^{t}} \left( \bm{\mathsf{y}} \mid \textcolor{orange}{\bm{\textsf{x}_{o}}} \right) &= \underbrace{\log \pi_{\bm{\theta}^{t+1}} \left( \bm{\mathsf{y}} \mid \textcolor{orange}{\bm{\mathsf{x}}_{o}} \right)}_{\mathbb{R}^{V \times L}} - \underbrace{\log \pi_{\bm{\theta}^{t}} \left( \bm{\mathsf{y}} \mid \textcolor{orange}{\bm{\mathsf{x}}_{o}} \right)}_{\mathbb{R}^{V \times L}} \\
    &= \underbrace{\nabla_{\bm{\theta}} \log \pi_{\bm{\theta}^{t}} \left( \bm{\mathsf{y}} \mid \textcolor{orange}{\bm{\mathsf{x}}_{o}} \right)}_{\mathbb{R}^{V \times L \times D}} \underbrace{\left( \bm{\theta}^{t+1} - \bm{\theta}^{t} \right)}_{\mathbb{R}^{D \times 1}} + \bigO \left( \left\| \bm{\theta}^{t+1} - \bm{\theta}^{t} \right\|^{2} \right) \\
\end{aligned}
\]
其中 \(D\) 是模型的参数量。
然后假设模型使用“更新样本(updating example)” \(\left(\textcolor{cyan}{\bm{\mathsf{x}}_{u}}, \textcolor{cyan}{\bm{\mathsf{y}}_{u}}\right)\),通过 SGD 完成参数更新。

借助 \(\log\) 函数的性质,我们可以将 \(L\) 个样本进行拆分:
\[
\begin{aligned}
    \underbrace{\nabla_{\bm{\theta}} \log \pi_{\bm{\theta}^{t}} \left( \bm{\mathsf{y}} \mid \textcolor{orange}{\bm{\mathsf{x}}_{o}} \right)}_{\mathbb{R}^{V \times L \times D}} &= \nabla_{\bm{\theta}} \log \prod_{l=1}^{L} \underbrace{\pi_{\bm{\theta}^{t}}^{(l)} \left( \bm{\mathsf{y}}^{(l)} \mid \textcolor{orange}{\bm{\mathsf{x}}_{o}^{(l)}} \right)}_{\mathbb{R}^{V \times D}} \\
    &= \sum_{l=1}^{L} \underbrace{ \nabla_{\bm{\theta}} \log \pi_{\bm{\theta}^{t}}^{(l)} \left( \bm{\mathsf{y}}^{(l)} \mid \textcolor{orange}{\bm{\mathsf{x}}_{o}^{(l)}} \right) }_{\mathbb{R}^{V \times D}}
\end{aligned}
\]

所以,对于 \(\nabla_{\bm{\theta}} \log \pi_{\bm{\theta}^{t}} \left( \bm{\mathsf{y}} \mid \textcolor{orange}{\bm{\mathsf{x}}_{o}} \right) \left( \bm{\theta}^{t+1} - \bm{\theta}^{t} \right)\) 这一项的评估,我们可以选取任意一个第 \(l\) 个样本所对应的项 \(\nabla_{\bm{\theta}} \log \pi_{\bm{\theta}^{t}}^{(l)} \left( \bm{\mathsf{y}}^{(l)} \mid \textcolor{orange}{\bm{\mathsf{x}}_{o}^{(l)}} \right)\),这就退化成了 \(L=1\) 时的场景。
\textcolor{magenta}{所以,我们在这里直接设 \(L=1\),忽略样本数量来进行后续分析。}
然后我们可以插入 SGD 的定义(可参考公式 (\ref{eq:GD-loss_iterative_update})),并重复利用链式法则:
\begin{equation}\label{eq:SGD-decomposition-detail}
\begin{aligned}
    &\underbrace{\nabla_{\bm{\theta}} \log \pi_{\bm{\theta}^{t}} \left( \bm{\mathsf{y}} \mid \textcolor{orange}{\bm{\mathsf{x}}_{o}} \right) \mid_{\bm{\theta}^{t}}}_{\mathbb{R}^{V \times D}} \cdot \underbrace{\left( \bm{\theta}^{t+1} - \bm{\theta}^{t} \right)}_{\mathbb{R}^{D \times 1}} \\
    &= \left( \underbrace{\nabla_{\bm{\mathsf{z}}} \log \pi_{\bm{\theta}^{t}} \left( \bm{\mathsf{y}} \mid \textcolor{orange}{\bm{\mathsf{x}}_{o}} \right) \mid_{\bm{\mathsf{z}}^{t}}}_{\mathbb{R}^{V \times V }} \cdot \underbrace{\nabla_{\bm{\theta}} \log \bm{\mathsf{z}}^{t} \left( \bm{\mathsf{y}} \mid \textcolor{orange}{\bm{\mathsf{x}}_{o}} \right) \mid_{\bm{\theta}^{t}}}_{\mathbb{R}^{V \times D}} \right) \cdot \left( - \eta \underbrace{\nabla_{\bm{\theta}} \mathcal{L} \left( f_{\bm{\theta}^{t}} \left( \textcolor{cyan}{\bm{\mathsf{x}}_{u}} \right), \textcolor{cyan}{\bm{\mathsf{y}}_{u}} \right) \mid_{\bm{\theta}^{t}}}_{\mathbb{R}^{1 \times D}}  \right)^{\top} \\
    &= \left( \underbrace{\nabla_{\bm{\mathsf{z}}} \log \pi_{\bm{\theta}^{t}} \left( \bm{\mathsf{y}} \mid \textcolor{orange}{\bm{\mathsf{x}}_{o}} \right) \mid_{\bm{\mathsf{z}}^{t}}}_{\mathbb{R}^{V \times V}} \cdot \underbrace{\nabla_{\bm{\theta}} \log \bm{\mathsf{z}}^{t} \left( \bm{\mathsf{y}} \mid \textcolor{orange}{\bm{\mathsf{x}}_{o}} \right) \mid_{\bm{\theta}^{t}}}_{\mathbb{R}^{V \times D}} \right) \cdot \left( - \eta \underbrace{\nabla_{\bm{\mathsf{z}}} \mathcal{L} \left( f_{\bm{\theta}^{t}} \left( \textcolor{cyan}{\bm{\mathsf{x}}_{u}} \right), \textcolor{cyan}{\bm{\mathsf{y}}_{u}} \right) \mid_{\bm{\mathsf{z}}^{t}}}_{\mathbb{R}^{1 \times V}} \cdot \underbrace{\nabla_{\bm{\theta}^{t}} \bm{\mathsf{z}}^{t} \left( f_{\bm{\theta}} \left( \textcolor{cyan}{\bm{\mathsf{x}}_{u}} \right), \textcolor{cyan}{\bm{\mathsf{y}}_{u}} \right) \mid_{\bm{\theta}^{t}}}_{\mathbb{R}^{V \times D}}  \right)^{\top} \\
     &= - \eta \underbrace{\nabla_{\bm{\mathsf{z}}} \log \pi_{\bm{\theta}^{t}} \left( \bm{\mathsf{y}} \mid \textcolor{orange}{\bm{\mathsf{x}}_{o}} \right) \mid_{\bm{\mathsf{z}}^{t}}}_{\mathbb{R}^{V \times V}} \cdot \left[ \underbrace{\nabla_{\bm{\theta}} \log \bm{\mathsf{z}}^{t} \left( \bm{\mathsf{y}} \mid \textcolor{orange}{\bm{\mathsf{x}}_{o}} \right) \mid_{\bm{\mathsf{z}}^{t}}}_{\mathbb{R}^{V \times D}} \cdot \underbrace{ \left( \nabla_{\bm{\theta}} \bm{\mathsf{z}}^{t} \left( f_{\bm{\theta}^{t}} \left( \textcolor{cyan}{\bm{\mathsf{x}}_{u}} \right), \textcolor{cyan}{\bm{\mathsf{y}}_{u}} \right) \mid_{\bm{\theta}^{t}} \right)^{\top}}_{\mathbb{R}^{D \times V}} \right] \cdot \underbrace{ \left[ \nabla_{\bm{\mathsf{z}}} \mathcal{L} \left( f_{\bm{\theta}^{t}} \left( \textcolor{cyan}{\bm{\mathsf{x}}_{u}} \right), \textcolor{cyan}{\bm{\mathsf{y}}_{u}} \right) \mid_{\bm{\mathsf{z}}^{t}} \right]^{\top}}_{\mathbb{R}^{V \times 1}} \\
     &= - \eta \mathcal{A}^{t} \left( \textcolor{orange}{\bm{\mathsf{x}}_{o}} \right) \mathcal{K}^{t} \left( \textcolor{orange}{\bm{\mathsf{x}}_{o}}, \textcolor{cyan}{\bm{\mathsf{x}}_{u}} \right) \mathcal{G}^{t} \left( \textcolor{cyan}{\bm{\mathsf{x}}_{u}}, \textcolor{cyan}{\bm{\mathsf{y}}_{u}} \right)
\end{aligned}
\end{equation}
对于公式 (\ref{eq:SGD-decomposition-detail}),我们需要注意:
\begin{itemize}
    \item \(\nabla_{\bm{\mathsf{z}}} \log \pi_{\bm{\theta}^{t}} \left( \bm{\mathsf{y}} \mid \textcolor{orange}{\bm{\mathsf{x}}_{o}} \right) \mid_{\bm{\mathsf{z}}^{t}}\) 描述模型对于输入 \(\textcolor{orange}{\bm{\mathsf{x}}_{o}}\) 的预测标签 \(\bm{\mathsf{y}}\) 的置信度变化,\(\bm{\mathsf{y}}\) 是模型对所有可能标签的预测结果(\(\textcolor{orange}{\bm{\mathsf{x}}_{o}}\) 暂时没有对应的真实标签,也没参与更新模型权重),可以认为这里是一个占位符,该部分只有一个自变量 \(\textcolor{orange}{\bm{\mathsf{x}}_{o}}\),所以我们可以使用\(\bm{\pi}^{t} \left( \textcolor{orange}{\bm{\mathsf{x}}_{o}} \right)\) 替代 \(\pi_{\bm{\theta}^{t}} \left( \bm{\mathsf{y}} \mid \textcolor{orange}{\bm{\mathsf{x}}_{o}} \right)\)。
    \item \(\nabla_{\bm{\theta}} \log \bm{\mathsf{z}}^{t} \left( \bm{\mathsf{y}} \mid \textcolor{orange}{\bm{\mathsf{x}}_{o}} \right) \mid_{\bm{\mathsf{z}}^{t}}\) 描述当我们调整模型参数 \(\bm{\theta}\) 时,模型的输出的 logits 矩阵会发生什么样的变化,同样这里只有一个有用的自变量 \(\textcolor{orange}{\bm{\mathsf{x}}_{o}}\)。所以我们用 \(\bm{\mathsf{z}}^{t} \left( \textcolor{orange}{\bm{\mathsf{x}}_{o}} \right) \) 代替 \(\bm{\mathsf{z}}^{t} \left( \bm{\mathsf{y}} \mid \textcolor{orange}{\bm{\mathsf{x}}_{o}} \right)\)
    \item \(\left( \nabla_{\bm{\theta}} \bm{\mathsf{z}}^{t} \left( f_{\bm{\theta}^{t}} \left( \textcolor{cyan}{\bm{\mathsf{x}}_{u}} \right), \textcolor{cyan}{\bm{\mathsf{y}}_{u}} \right) \mid_{\bm{\theta}^{t}} \right)^{\top}\) 这部分描述模型前向传播中,调整参数 \(\bm{\theta}\),如何影响最终 logits 矩阵的输出。这部分只依赖参数 \(\bm{\theta}\) 和 输入 \(\textcolor{cyan}{\bm{\mathsf{x}}_{u}}\),与真实标签 \(\textcolor{cyan}{\bm{\mathsf{y}}_{u}}\) 无关,同样这里只有一个有用的自变量 \(\textcolor{cyan}{\bm{\mathsf{x}}_{u}}\)。所以我们用 \(\bm{\mathsf{z}}^{t} \left( \textcolor{cyan}{\bm{\mathsf{x}}_{u}} \right)\) 代替 \(\bm{\mathsf{z}}^{t} \left( f_{\bm{\theta}^{t}} \left( \textcolor{cyan}{\bm{\mathsf{x}}_{u}} \right), \textcolor{cyan}{\bm{\mathsf{y}}_{u}} \right)\)
    \item \(\left[ \nabla_{\bm{\mathsf{z}}} \mathcal{L} \left( f_{\bm{\theta}^{t}} \left( \textcolor{cyan}{\bm{\mathsf{x}}_{u}} \right), \textcolor{cyan}{\bm{\mathsf{y}}_{u}} \right) \mid_{\bm{\mathsf{z}}^{t}} \right]^{\top}\) 这部分描述损失函数 \(\mathcal{L} \left( f_{\bm{\theta}^{t}} \left( \textcolor{cyan}{\bm{\mathsf{x}}_{u}} \right), \textcolor{cyan}{\bm{\mathsf{y}}_{u}} \right)\) 如何受到输出 \(\bm{\mathsf{z}}^{t}\) 的影响。损失函数的计算中,需要通过对输入 \(\textcolor{cyan}{\bm{\mathsf{x}}_{u}}\) 进行预测(前向传播)得到预测标签 \(\textcolor{cyan}{\hat{\bm{\mathsf{y}}}_{u}}\),然后与真实标签 \(\textcolor{cyan}{\bm{\mathsf{y}}_{u}}\) 一起计算交叉熵。所以,我们用 \(\mathcal{L} \left( \textcolor{cyan}{\bm{\mathsf{x}}_{u}}, \textcolor{cyan}{\bm{\mathsf{y}}_{u}} \right)\) 代替 \(\mathcal{L} \left( f_{\bm{\theta}^{t}} \left( \textcolor{cyan}{\bm{\mathsf{x}}_{u}} \right), \textcolor{cyan}{\bm{\mathsf{y}}_{u}} \right)\)。
\end{itemize}
所以,公式 (\ref{eq:SGD-decomposition}) 可以被写为:
\begin{equation}\label{eq:SGD-decomposition-simplified}
\begin{aligned}
    &\underbrace{\nabla_{\bm{\theta}} \log \pi_{\bm{\theta}^{t}} \left( \bm{\mathsf{y}} \mid \textcolor{orange}{\bm{\mathsf{x}}_{o}} \right) \mid_{\bm{\theta}^{t}}}_{\mathbb{R}^{V \times D}} \cdot \underbrace{\left( \bm{\theta}^{t+1} - \bm{\theta}^{t} \right)}_{\mathbb{R}^{D \times 1}} \\
    &= - \eta \underbrace{\nabla_{\bm{\mathsf{z}}} \log \bm{\pi}^{t} \left( \textcolor{orange}{\bm{\mathsf{x}}_{o}} \right) \mid_{\bm{\mathsf{z}}^{t}}}_{\mathbb{R}^{V \times V}} \cdot \left[ \underbrace{\nabla_{\bm{\theta}} \log \bm{\mathsf{z}}^{t} \left( \textcolor{orange}{\bm{\mathsf{x}}_{o}} \right) \mid_{\bm{\mathsf{z}}^{t}}}_{\mathbb{R}^{V \times D}} \cdot \underbrace{ \left( \nabla_{\bm{\theta}} \bm{\mathsf{z}}^{t} \left( \textcolor{cyan}{\bm{\mathsf{x}}_{u}} \right) \mid_{\bm{\theta}^{t}} \right)^{\top}}_{\mathbb{R}^{D \times V}} \right] \cdot \underbrace{ \left[ \nabla_{\bm{\mathsf{z}}} \mathcal{L} \left( \textcolor{cyan}{\bm{\mathsf{x}}_{u}}, \textcolor{cyan}{\bm{\mathsf{y}}_{u}} \right) \mid_{\bm{\mathsf{z}}^{t}} \right]^{\top}}_{\mathbb{R}^{V \times 1}} \\
    &= - \eta \mathcal{A}^{t} \left( \textcolor{orange}{\bm{\mathsf{x}}_{o}} \right) \mathcal{K}^{t} \left( \textcolor{orange}{\bm{\mathsf{x}}_{o}}, \textcolor{cyan}{\bm{\mathsf{x}}_{u}} \right) \mathcal{G}^{t} \left( \textcolor{cyan}{\bm{\mathsf{x}}_{u}}, \textcolor{cyan}{\bm{\mathsf{y}}_{u}} \right)
\end{aligned}
\end{equation}
此处 \( \mathcal{A}^{t} \left( \textcolor{orange}{\bm{\mathsf{x}}_{o}} \right) = \nabla_{\bm{\mathsf{z}}} \log \bm{\pi}^{t} \left( \textcolor{orange}{\bm{\mathsf{x}}_{o}} \right) \mid_{\bm{\mathsf{z}}^{t}} \),\(\mathcal{K}^{t} \left( \textcolor{orange}{\bm{\mathsf{x}}_{o}}, \textcolor{cyan}{\bm{\mathsf{x}}_{u}} \right) = \nabla_{\bm{\theta}} \log \bm{\mathsf{z}}^{t} \left( \textcolor{orange}{\bm{\mathsf{x}}_{o}} \right) \mid_{\bm{\mathsf{z}}^{t}} \cdot \left( \nabla_{\bm{\theta}} \bm{\mathsf{z}}^{t} \left( \textcolor{cyan}{\bm{\mathsf{x}}_{u}} \right) \mid_{\bm{\theta}^{t}} \right)^{\top}\),\(\mathcal{G}^{t} \left( \textcolor{cyan}{\bm{\mathsf{x}}_{u}}, \textcolor{cyan}{\bm{\mathsf{y}}_{u}} \right) = \left[ \nabla_{\bm{\mathsf{z}}} \mathcal{L} \left( \textcolor{cyan}{\bm{\mathsf{x}}_{u}}, \textcolor{cyan}{\bm{\mathsf{y}}_{u}} \right) \mid_{\bm{\mathsf{z}}^{t}} \right]^{\top}\)。

其中,我们可以利用上述的符号来表述参数 \(\bm{\theta}\) 的变化:
\begin{equation}\label{eq:SGD-learning_dynamic_theta_decomposition}
    \bm{\theta}^{t+1} - \bm{\theta}^{t} = - \eta \left[ \nabla_{\bm{\theta}} \bm{\mathsf{z}}^{t} \left( \textcolor{cyan}{\bm{\mathsf{x}}_{u}} \right) \mid_{\bm{\theta}^{t}}  \right]^{\top} \mathcal{G}^{t} \left( \textcolor{cyan}{\bm{\mathsf{x}}_{u}}, \textcolor{cyan}{\bm{\mathsf{y}}_{u}} \right)
\end{equation}
并且将公式 (\ref{eq:SGD-learning_dynamic_theta_decomposition}) 代入上述的高阶项目,我们得到:
\[
    \bigO \left( \left\| \bm{\theta}^{t+1} - \bm{\theta}^{t} \right\|^{2} \right) = \bigO \left( \eta^{2} \cdot \left\| \left[ \nabla_{\bm{\theta}} \bm{\mathsf{z}}^{t} \left( \textcolor{cyan}{\bm{\mathsf{x}}_{u}} \right) \mid_{\bm{\theta}^{t}}  \right]^{\top} \right\|^{2}_{\texttt{op}} \cdot \left\| \mathcal{G}^{t} \left( \textcolor{cyan}{\bm{\mathsf{x}}_{u}}, \textcolor{cyan}{\bm{\mathsf{y}}_{u}} \right) \right\|^{2}_{\texttt{op}}\right)
\]
这里残差项 \(\mathcal{G}^{t} \left( \textcolor{cyan}{\bm{\mathsf{x}}_{u}}, \textcolor{cyan}{\bm{\mathsf{y}}_{u}} \right)\) 通常是有界的(参考引理~\ref{lemma:bounded_loss_gradient}),所以我们得到~\footnote{矩阵和其转置矩阵的奇异值是相同的,换句话说,\(\mathbf{A}\) 和 \(\mathbf{A}^{\top}\) 的奇异值集合是一样的。所以它们的谱范数是一样的,即 \( \left\| \mathbf{A} \right\|^{2}_{\texttt{op}} = \left\| \mathbf{A}^{\top} \right\|^{2}_{\texttt{op}} \)}:
\[
    \bigO \left( \left\| \bm{\theta}^{t+1} - \bm{\theta}^{t} \right\|^{2} \right) = \bigO \left( \eta^{2} \left\| \left[ \nabla_{\bm{\theta}} \bm{\mathsf{z}}^{t} \left( \textcolor{cyan}{\bm{\mathsf{x}}_{u}} \right) \mid_{\bm{\theta}^{t}}  \right]^{\top} \right\|^{2}_{\texttt{op}} \right)
\]
\end{proof}




\nocite{*}

\printbibliography[heading=bibintoc, title=\ebibname]
\appendix

\chapter{文中用到的一些基本引理}

\begin{lemma}\label{lemma:bounded_loss_gradient}
    loss 函数对 logits 矩阵的梯度通常是有界。
\end{lemma}

\begin{proof}[引理~\ref{lemma:bounded_loss_gradient}]
    例如,对于分类问题 或者 大语言模型(LLM)的训练中,交叉熵(Cross-Entropy)是最常用的损失函数。
参考 \texttt{Softmax} (公式~\ref{eq:softmax})和 交叉熵 的公式(公式~\ref{eq:cross_entropy_loss})。
\begin{equation}\label{eq:cross_entropy_loss-gradient}
\begin{aligned}
\nabla_{\bm{\mathsf{z}}} \mathcal{L} \left( f_{\bm{\theta}} \left( \bm{\mathsf{X}} \right), \bm{\mathsf{Y}} \right) &= - \frac{1}{N} \sum_{i=1}^{N} \frac{\partial \sum_{j=1}^{L} y^{(i)}_{j} \log \hat{y}^{(i)}_{j} }{\partial \bm{\mathsf{z}}^{(i)}} \\
    &= - \frac{1}{N} \sum_{i=1}^{N} \sum_{k=1}^{V} \frac{\partial \sum_{j=1}^{L} y^{(i)}_{j} \log \hat{y}^{(i)}_{j} }{\partial z^{(i)}_{k}} \\
    &= - \frac{1}{N} \sum_{i=1}^{N} \sum_{k=1}^{V} \left( y^{(i)}_{\texttt{ct}} \cdot \frac{1}{\hat{y}^{(i)}_{\texttt{ct}}} \cdot \frac{\partial \hat{y}^{(i)}_{\texttt{ct}}}{\partial z^{(i)}_{k}} + \sum_{j \neq \texttt{ct}} y^{(i)}_{j} \cdot \frac{1}{\hat{y}^{(i)}_{j}} \cdot \frac{\partial \hat{y}^{(i)}_{j}}{\partial z^{(i)}_{k}} \right) \\
\end{aligned}
\end{equation}
其中 \(\texttt{ct}\) 是正确(correct)的类别的索引。
我们知道 \texttt{Softmax} 对 logits 矩阵的分量 \(z^{(i)}_{k}\) 的梯度为:
\[
    \frac{\partial \hat{y}^{(i)}_{j}}{\partial z^{(i)}_{k}} = \hat{y}^{(i)}_{j} \left( \delta_{j,k} - \hat{y}^{(i)}_{k} \right)
\]
% \(\delta_{j,k}\) 是 Kronecker delta 函数,当 \(j=k\) 时为 \(1\),否则为 \(0\)。
公式 (\ref{eq:cross_entropy_loss-gradient}) 中的梯度可以进一步简化为:
\begin{equation}\label{eq:cross_entropy_loss-gradient_simplified}
\begin{aligned}
\frac{\partial \mathcal{L} \left( f_{\bm{\theta}} \left( \bm{\mathsf{X}} \right), \bm{\mathsf{Y}} \right)}{\partial z_{k}^{(i)}}  &= - \sum_{j=1}^{V} y^{(i)}_{j} \cdot \frac{1}{\hat{y}^{(i)}_{j}} \cdot \hat{y}^{(i)}_{j} \left( \delta_{j,k} - \hat{y}^{(i)}_{k} \right) \\
&= \hat{y}^{(i)}_{k} - y^{(i)}_{k}
\end{aligned}
\end{equation}
注意,这里利用了 one-hot 编码的性质(只在一个量上为 \(1\),其他量全为 \(0\)),所以只保留了 \(y^{(i)}_{\texttt{ct}}\) 这一项,消除了一个 \(\sum_{k=1}^{V}\)。

对梯度的每一个维度,因为 \texttt{Softmax} 函数的性质:\(\hat{y}^{(i)}_{k} \in \left( 0, 1 \right)\);对于正确类别,\(\frac{\partial \mathcal{L}}{\partial z_{k}^{(i)}} = \hat{y}^{(i)}_{k} - 1 \in \left(-1, 0\right)\),对于错误的类别,\(\frac{\partial \mathcal{L}}{\partial z_{k}^{(i)}} = \hat{y}^{(i)}_{k} - 0 \in \left( 0, 1 \right)\)。
也就是说上界是 \(1\),下界是 \(- 1\),所以损失函数对于 logits 矩阵的梯度是有界的。
\end{proof}

\end{document}
